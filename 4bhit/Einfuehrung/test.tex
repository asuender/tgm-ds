% Options for packages loaded elsewhere
\PassOptionsToPackage{unicode}{hyperref}
\PassOptionsToPackage{hyphens}{url}
%
\documentclass[
]{article}
\usepackage{amsmath,amssymb}
\usepackage{lmodern}
\usepackage{iftex}
\ifPDFTeX
  \usepackage[T1]{fontenc}
  \usepackage[utf8]{inputenc}
  \usepackage{textcomp} % provide euro and other symbols
\else % if luatex or xetex
  \usepackage{unicode-math}
  \defaultfontfeatures{Scale=MatchLowercase}
  \defaultfontfeatures[\rmfamily]{Ligatures=TeX,Scale=1}
\fi
% Use upquote if available, for straight quotes in verbatim environments
\IfFileExists{upquote.sty}{\usepackage{upquote}}{}
\IfFileExists{microtype.sty}{% use microtype if available
  \usepackage[]{microtype}
  \UseMicrotypeSet[protrusion]{basicmath} % disable protrusion for tt fonts
}{}
\makeatletter
\@ifundefined{KOMAClassName}{% if non-KOMA class
  \IfFileExists{parskip.sty}{%
    \usepackage{parskip}
  }{% else
    \setlength{\parindent}{0pt}
    \setlength{\parskip}{6pt plus 2pt minus 1pt}}
}{% if KOMA class
  \KOMAoptions{parskip=half}}
\makeatother
\usepackage{xcolor}
\usepackage[margin=1in]{geometry}
\usepackage{color}
\usepackage{fancyvrb}
\newcommand{\VerbBar}{|}
\newcommand{\VERB}{\Verb[commandchars=\\\{\}]}
\DefineVerbatimEnvironment{Highlighting}{Verbatim}{commandchars=\\\{\}}
% Add ',fontsize=\small' for more characters per line
\usepackage{framed}
\definecolor{shadecolor}{RGB}{248,248,248}
\newenvironment{Shaded}{\begin{snugshade}}{\end{snugshade}}
\newcommand{\AlertTok}[1]{\textcolor[rgb]{0.94,0.16,0.16}{#1}}
\newcommand{\AnnotationTok}[1]{\textcolor[rgb]{0.56,0.35,0.01}{\textbf{\textit{#1}}}}
\newcommand{\AttributeTok}[1]{\textcolor[rgb]{0.77,0.63,0.00}{#1}}
\newcommand{\BaseNTok}[1]{\textcolor[rgb]{0.00,0.00,0.81}{#1}}
\newcommand{\BuiltInTok}[1]{#1}
\newcommand{\CharTok}[1]{\textcolor[rgb]{0.31,0.60,0.02}{#1}}
\newcommand{\CommentTok}[1]{\textcolor[rgb]{0.56,0.35,0.01}{\textit{#1}}}
\newcommand{\CommentVarTok}[1]{\textcolor[rgb]{0.56,0.35,0.01}{\textbf{\textit{#1}}}}
\newcommand{\ConstantTok}[1]{\textcolor[rgb]{0.00,0.00,0.00}{#1}}
\newcommand{\ControlFlowTok}[1]{\textcolor[rgb]{0.13,0.29,0.53}{\textbf{#1}}}
\newcommand{\DataTypeTok}[1]{\textcolor[rgb]{0.13,0.29,0.53}{#1}}
\newcommand{\DecValTok}[1]{\textcolor[rgb]{0.00,0.00,0.81}{#1}}
\newcommand{\DocumentationTok}[1]{\textcolor[rgb]{0.56,0.35,0.01}{\textbf{\textit{#1}}}}
\newcommand{\ErrorTok}[1]{\textcolor[rgb]{0.64,0.00,0.00}{\textbf{#1}}}
\newcommand{\ExtensionTok}[1]{#1}
\newcommand{\FloatTok}[1]{\textcolor[rgb]{0.00,0.00,0.81}{#1}}
\newcommand{\FunctionTok}[1]{\textcolor[rgb]{0.00,0.00,0.00}{#1}}
\newcommand{\ImportTok}[1]{#1}
\newcommand{\InformationTok}[1]{\textcolor[rgb]{0.56,0.35,0.01}{\textbf{\textit{#1}}}}
\newcommand{\KeywordTok}[1]{\textcolor[rgb]{0.13,0.29,0.53}{\textbf{#1}}}
\newcommand{\NormalTok}[1]{#1}
\newcommand{\OperatorTok}[1]{\textcolor[rgb]{0.81,0.36,0.00}{\textbf{#1}}}
\newcommand{\OtherTok}[1]{\textcolor[rgb]{0.56,0.35,0.01}{#1}}
\newcommand{\PreprocessorTok}[1]{\textcolor[rgb]{0.56,0.35,0.01}{\textit{#1}}}
\newcommand{\RegionMarkerTok}[1]{#1}
\newcommand{\SpecialCharTok}[1]{\textcolor[rgb]{0.00,0.00,0.00}{#1}}
\newcommand{\SpecialStringTok}[1]{\textcolor[rgb]{0.31,0.60,0.02}{#1}}
\newcommand{\StringTok}[1]{\textcolor[rgb]{0.31,0.60,0.02}{#1}}
\newcommand{\VariableTok}[1]{\textcolor[rgb]{0.00,0.00,0.00}{#1}}
\newcommand{\VerbatimStringTok}[1]{\textcolor[rgb]{0.31,0.60,0.02}{#1}}
\newcommand{\WarningTok}[1]{\textcolor[rgb]{0.56,0.35,0.01}{\textbf{\textit{#1}}}}
\usepackage{graphicx}
\makeatletter
\def\maxwidth{\ifdim\Gin@nat@width>\linewidth\linewidth\else\Gin@nat@width\fi}
\def\maxheight{\ifdim\Gin@nat@height>\textheight\textheight\else\Gin@nat@height\fi}
\makeatother
% Scale images if necessary, so that they will not overflow the page
% margins by default, and it is still possible to overwrite the defaults
% using explicit options in \includegraphics[width, height, ...]{}
\setkeys{Gin}{width=\maxwidth,height=\maxheight,keepaspectratio}
% Set default figure placement to htbp
\makeatletter
\def\fps@figure{htbp}
\makeatother
\setlength{\emergencystretch}{3em} % prevent overfull lines
\providecommand{\tightlist}{%
  \setlength{\itemsep}{0pt}\setlength{\parskip}{0pt}}
\setcounter{secnumdepth}{-\maxdimen} % remove section numbering
\ifLuaTeX
  \usepackage{selnolig}  % disable illegal ligatures
\fi
\IfFileExists{bookmark.sty}{\usepackage{bookmark}}{\usepackage{hyperref}}
\IfFileExists{xurl.sty}{\usepackage{xurl}}{} % add URL line breaks if available
\urlstyle{same} % disable monospaced font for URLs
\hypersetup{
  pdftitle={Testfile},
  pdfauthor={Alexandra Posekany},
  hidelinks,
  pdfcreator={LaTeX via pandoc}}

\title{Testfile}
\author{Alexandra Posekany}
\date{2022-09-20}

\begin{document}
\maketitle

\hypertarget{r-markdown}{%
\subsection{R Markdown}\label{r-markdown}}

This is an R Markdown document. Markdown is a simple formatting syntax
for authoring HTML, PDF, and MS Word documents. For more details on
using R Markdown see \url{http://rmarkdown.rstudio.com}.

When you click the \textbf{Knit} \emph{button} a document will be
generated that includes both content as well as the output of any
embedded R code chunks within the document. You can embed an R code
chunk like this:

\begin{Shaded}
\begin{Highlighting}[]
\FunctionTok{summary}\NormalTok{(cars)}
\end{Highlighting}
\end{Shaded}

\begin{verbatim}
##      speed           dist       
##  Min.   : 4.0   Min.   :  2.00  
##  1st Qu.:12.0   1st Qu.: 26.00  
##  Median :15.0   Median : 36.00  
##  Mean   :15.4   Mean   : 42.98  
##  3rd Qu.:19.0   3rd Qu.: 56.00  
##  Max.   :25.0   Max.   :120.00
\end{verbatim}

\begin{Shaded}
\begin{Highlighting}[]
\NormalTok{cars\_adaptiert }\OtherTok{\textless{}{-}}\NormalTok{ cars}
\NormalTok{cars\_adaptiert}\SpecialCharTok{$}\NormalTok{speedkmh }\OtherTok{\textless{}{-}}\NormalTok{ cars\_adaptiert}\SpecialCharTok{$}\NormalTok{speed }\SpecialCharTok{*} \FloatTok{1.609344}
\FunctionTok{str}\NormalTok{(cars\_adaptiert)}
\end{Highlighting}
\end{Shaded}

\begin{verbatim}
## 'data.frame':    50 obs. of  3 variables:
##  $ speed   : num  4 4 7 7 8 9 10 10 10 11 ...
##  $ dist    : num  2 10 4 22 16 10 18 26 34 17 ...
##  $ speedkmh: num  6.44 6.44 11.27 11.27 12.87 ...
\end{verbatim}

\begin{Shaded}
\begin{Highlighting}[]
\NormalTok{x }\OtherTok{\textless{}{-}} \FunctionTok{c}\NormalTok{(}\DecValTok{2}\NormalTok{,}\DecValTok{3}\NormalTok{,}\DecValTok{4}\NormalTok{)}
\FunctionTok{str}\NormalTok{(x)}
\end{Highlighting}
\end{Shaded}

\begin{verbatim}
##  num [1:3] 2 3 4
\end{verbatim}

\begin{Shaded}
\begin{Highlighting}[]
\NormalTok{x }\OtherTok{\textless{}{-}} \FunctionTok{c}\NormalTok{(}\DecValTok{2}\NormalTok{,}\StringTok{"3"}\NormalTok{,}\DecValTok{4}\NormalTok{)}
\FunctionTok{str}\NormalTok{(x)}
\end{Highlighting}
\end{Shaded}

\begin{verbatim}
##  chr [1:3] "2" "3" "4"
\end{verbatim}

\begin{Shaded}
\begin{Highlighting}[]
\NormalTok{y}\OtherTok{\textless{}{-}}\FunctionTok{c}\NormalTok{(}\ConstantTok{TRUE}\NormalTok{,}\ConstantTok{FALSE}\NormalTok{,}\ConstantTok{TRUE}\NormalTok{)}
\FunctionTok{str}\NormalTok{(y)}
\end{Highlighting}
\end{Shaded}

\begin{verbatim}
##  logi [1:3] TRUE FALSE TRUE
\end{verbatim}

\begin{Shaded}
\begin{Highlighting}[]
\NormalTok{listen }\OtherTok{\textless{}{-}} \FunctionTok{list}\NormalTok{(x,y,cars)}
\FunctionTok{str}\NormalTok{(listen)}
\end{Highlighting}
\end{Shaded}

\begin{verbatim}
## List of 3
##  $ : chr [1:3] "2" "3" "4"
##  $ : logi [1:3] TRUE FALSE TRUE
##  $ :'data.frame':    50 obs. of  2 variables:
##   ..$ speed: num [1:50] 4 4 7 7 8 9 10 10 10 11 ...
##   ..$ dist : num [1:50] 2 10 4 22 16 10 18 26 34 17 ...
\end{verbatim}

\begin{Shaded}
\begin{Highlighting}[]
\NormalTok{(listen[[}\DecValTok{3}\NormalTok{]])[}\DecValTok{1}\SpecialCharTok{:}\DecValTok{10}\NormalTok{,]}
\end{Highlighting}
\end{Shaded}

\begin{verbatim}
##    speed dist
## 1      4    2
## 2      4   10
## 3      7    4
## 4      7   22
## 5      8   16
## 6      9   10
## 7     10   18
## 8     10   26
## 9     10   34
## 10    11   17
\end{verbatim}

\begin{Shaded}
\begin{Highlighting}[]
\NormalTok{schueler }\OtherTok{\textless{}{-}} \FunctionTok{data.frame}\NormalTok{(}\AttributeTok{alter =} \FunctionTok{c}\NormalTok{(}\DecValTok{17}\NormalTok{,}\DecValTok{18}\NormalTok{,}\DecValTok{17}\NormalTok{,}\DecValTok{19}\NormalTok{,}\DecValTok{21}\NormalTok{), }\AttributeTok{geschlecht =} \FunctionTok{c}\NormalTok{(}\StringTok{"F"}\NormalTok{,}\StringTok{"M"}\NormalTok{,}\StringTok{"M"}\NormalTok{,}\StringTok{"M"}\NormalTok{,}\StringTok{"M"}\NormalTok{))}
\FunctionTok{str}\NormalTok{(schueler)}
\end{Highlighting}
\end{Shaded}

\begin{verbatim}
## 'data.frame':    5 obs. of  2 variables:
##  $ alter     : num  17 18 17 19 21
##  $ geschlecht: chr  "F" "M" "M" "M" ...
\end{verbatim}

\begin{Shaded}
\begin{Highlighting}[]
\NormalTok{schueler}\SpecialCharTok{$}\NormalTok{geschlecht }\OtherTok{\textless{}{-}} \FunctionTok{factor}\NormalTok{(schueler}\SpecialCharTok{$}\NormalTok{geschlecht)}
\FunctionTok{str}\NormalTok{(schueler)}
\end{Highlighting}
\end{Shaded}

\begin{verbatim}
## 'data.frame':    5 obs. of  2 variables:
##  $ alter     : num  17 18 17 19 21
##  $ geschlecht: Factor w/ 2 levels "F","M": 1 2 2 2 2
\end{verbatim}

\begin{Shaded}
\begin{Highlighting}[]
\FunctionTok{summary}\NormalTok{(schueler)}
\end{Highlighting}
\end{Shaded}

\begin{verbatim}
##      alter      geschlecht
##  Min.   :17.0   F:1       
##  1st Qu.:17.0   M:4       
##  Median :18.0             
##  Mean   :18.4             
##  3rd Qu.:19.0             
##  Max.   :21.0
\end{verbatim}

\begin{table}[ht]
\centering
\begin{tabular}{lll}
  \hline
 Speed (mph) & Stopping distance (ft) &  Speed (kmh) \\ 
  \hline
Min.   : 4.0   & Min.   :  2.00   & Min.   : 6.437   \\ 
  1st Qu.:12.0   & 1st Qu.: 26.00   & 1st Qu.:19.312   \\ 
  Median :15.0   & Median : 36.00   & Median :24.140   \\ 
  Mean   :15.4   & Mean   : 42.98   & Mean   :24.784   \\ 
  3rd Qu.:19.0   & 3rd Qu.: 56.00   & 3rd Qu.:30.578   \\ 
  Max.   :25.0   & Max.   :120.00   & Max.   :40.234   \\ 
   \hline
\end{tabular}
\caption{Zusammenfassung von cars; 1. Spalte speed in mph; 2. Spalte stopping distance in ft; 3. Spalte speed in km/h} 
\end{table}

Hier unten sind Beschreibungen.

\[
\int_{-3}^{4}{e^{-sin(x)} dx}
\]

\hypertarget{including-plots}{%
\subsection{Including Plots}\label{including-plots}}

You can also embed plots, for example:

\begin{Shaded}
\begin{Highlighting}[]
\FunctionTok{plot}\NormalTok{(pressure)}
\end{Highlighting}
\end{Shaded}

\includegraphics{test_files/figure-latex/pressure2-1.pdf}

\includegraphics{test_files/figure-latex/unnamed-chunk-2-1.pdf}

\includegraphics{test_files/figure-latex/unnamed-chunk-3-1.pdf}

\includegraphics{test_files/figure-latex/unnamed-chunk-4-1.pdf}

Note that the \texttt{echo\ =\ FALSE} parameter was added to the code
chunk to prevent printing of the R code that generated the plot.

\end{document}
