% Options for packages loaded elsewhere
\PassOptionsToPackage{unicode}{hyperref}
\PassOptionsToPackage{hyphens}{url}
%
\documentclass[
]{article}
\usepackage{amsmath,amssymb}
\usepackage{iftex}
\ifPDFTeX
  \usepackage[T1]{fontenc}
  \usepackage[utf8]{inputenc}
  \usepackage{textcomp} % provide euro and other symbols
\else % if luatex or xetex
  \usepackage{unicode-math} % this also loads fontspec
  \defaultfontfeatures{Scale=MatchLowercase}
  \defaultfontfeatures[\rmfamily]{Ligatures=TeX,Scale=1}
\fi
\usepackage{lmodern}
\ifPDFTeX\else
  % xetex/luatex font selection
\fi
% Use upquote if available, for straight quotes in verbatim environments
\IfFileExists{upquote.sty}{\usepackage{upquote}}{}
\IfFileExists{microtype.sty}{% use microtype if available
  \usepackage[]{microtype}
  \UseMicrotypeSet[protrusion]{basicmath} % disable protrusion for tt fonts
}{}
\makeatletter
\@ifundefined{KOMAClassName}{% if non-KOMA class
  \IfFileExists{parskip.sty}{%
    \usepackage{parskip}
  }{% else
    \setlength{\parindent}{0pt}
    \setlength{\parskip}{6pt plus 2pt minus 1pt}}
}{% if KOMA class
  \KOMAoptions{parskip=half}}
\makeatother
\usepackage{xcolor}
\usepackage[margin=1in]{geometry}
\usepackage{color}
\usepackage{fancyvrb}
\newcommand{\VerbBar}{|}
\newcommand{\VERB}{\Verb[commandchars=\\\{\}]}
\DefineVerbatimEnvironment{Highlighting}{Verbatim}{commandchars=\\\{\}}
% Add ',fontsize=\small' for more characters per line
\usepackage{framed}
\definecolor{shadecolor}{RGB}{248,248,248}
\newenvironment{Shaded}{\begin{snugshade}}{\end{snugshade}}
\newcommand{\AlertTok}[1]{\textcolor[rgb]{0.94,0.16,0.16}{#1}}
\newcommand{\AnnotationTok}[1]{\textcolor[rgb]{0.56,0.35,0.01}{\textbf{\textit{#1}}}}
\newcommand{\AttributeTok}[1]{\textcolor[rgb]{0.13,0.29,0.53}{#1}}
\newcommand{\BaseNTok}[1]{\textcolor[rgb]{0.00,0.00,0.81}{#1}}
\newcommand{\BuiltInTok}[1]{#1}
\newcommand{\CharTok}[1]{\textcolor[rgb]{0.31,0.60,0.02}{#1}}
\newcommand{\CommentTok}[1]{\textcolor[rgb]{0.56,0.35,0.01}{\textit{#1}}}
\newcommand{\CommentVarTok}[1]{\textcolor[rgb]{0.56,0.35,0.01}{\textbf{\textit{#1}}}}
\newcommand{\ConstantTok}[1]{\textcolor[rgb]{0.56,0.35,0.01}{#1}}
\newcommand{\ControlFlowTok}[1]{\textcolor[rgb]{0.13,0.29,0.53}{\textbf{#1}}}
\newcommand{\DataTypeTok}[1]{\textcolor[rgb]{0.13,0.29,0.53}{#1}}
\newcommand{\DecValTok}[1]{\textcolor[rgb]{0.00,0.00,0.81}{#1}}
\newcommand{\DocumentationTok}[1]{\textcolor[rgb]{0.56,0.35,0.01}{\textbf{\textit{#1}}}}
\newcommand{\ErrorTok}[1]{\textcolor[rgb]{0.64,0.00,0.00}{\textbf{#1}}}
\newcommand{\ExtensionTok}[1]{#1}
\newcommand{\FloatTok}[1]{\textcolor[rgb]{0.00,0.00,0.81}{#1}}
\newcommand{\FunctionTok}[1]{\textcolor[rgb]{0.13,0.29,0.53}{\textbf{#1}}}
\newcommand{\ImportTok}[1]{#1}
\newcommand{\InformationTok}[1]{\textcolor[rgb]{0.56,0.35,0.01}{\textbf{\textit{#1}}}}
\newcommand{\KeywordTok}[1]{\textcolor[rgb]{0.13,0.29,0.53}{\textbf{#1}}}
\newcommand{\NormalTok}[1]{#1}
\newcommand{\OperatorTok}[1]{\textcolor[rgb]{0.81,0.36,0.00}{\textbf{#1}}}
\newcommand{\OtherTok}[1]{\textcolor[rgb]{0.56,0.35,0.01}{#1}}
\newcommand{\PreprocessorTok}[1]{\textcolor[rgb]{0.56,0.35,0.01}{\textit{#1}}}
\newcommand{\RegionMarkerTok}[1]{#1}
\newcommand{\SpecialCharTok}[1]{\textcolor[rgb]{0.81,0.36,0.00}{\textbf{#1}}}
\newcommand{\SpecialStringTok}[1]{\textcolor[rgb]{0.31,0.60,0.02}{#1}}
\newcommand{\StringTok}[1]{\textcolor[rgb]{0.31,0.60,0.02}{#1}}
\newcommand{\VariableTok}[1]{\textcolor[rgb]{0.00,0.00,0.00}{#1}}
\newcommand{\VerbatimStringTok}[1]{\textcolor[rgb]{0.31,0.60,0.02}{#1}}
\newcommand{\WarningTok}[1]{\textcolor[rgb]{0.56,0.35,0.01}{\textbf{\textit{#1}}}}
\usepackage{longtable,booktabs,array}
\usepackage{calc} % for calculating minipage widths
% Correct order of tables after \paragraph or \subparagraph
\usepackage{etoolbox}
\makeatletter
\patchcmd\longtable{\par}{\if@noskipsec\mbox{}\fi\par}{}{}
\makeatother
% Allow footnotes in longtable head/foot
\IfFileExists{footnotehyper.sty}{\usepackage{footnotehyper}}{\usepackage{footnote}}
\makesavenoteenv{longtable}
\usepackage{graphicx}
\makeatletter
\def\maxwidth{\ifdim\Gin@nat@width>\linewidth\linewidth\else\Gin@nat@width\fi}
\def\maxheight{\ifdim\Gin@nat@height>\textheight\textheight\else\Gin@nat@height\fi}
\makeatother
% Scale images if necessary, so that they will not overflow the page
% margins by default, and it is still possible to overwrite the defaults
% using explicit options in \includegraphics[width, height, ...]{}
\setkeys{Gin}{width=\maxwidth,height=\maxheight,keepaspectratio}
% Set default figure placement to htbp
\makeatletter
\def\fps@figure{htbp}
\makeatother
\setlength{\emergencystretch}{3em} % prevent overfull lines
\providecommand{\tightlist}{%
  \setlength{\itemsep}{0pt}\setlength{\parskip}{0pt}}
\setcounter{secnumdepth}{5}
\usepackage{lipsum}  
\usepackage{nicefrac}
\usepackage{amsmath}
\usepackage{graphicx}
\usepackage[export]{adjustbox}
\usepackage{nonfloat}
\usepackage{multicol}

\usepackage[most]{tcolorbox}

\usepackage{mathptmx}
\usepackage[T1]{fontenc}

\definecolor{cnorm}{rgb}{0.25, 0.25, 0.25}
\definecolor{cdef}{rgb}{0.85, 0.65, 0.15}
\definecolor{cbsp}{rgb}{0.0, 0.5, 0.0}
\definecolor{caufg}{rgb}{0.0, 0.0, 0.5}
\definecolor{cmaxima}{rgb}{0.95, 0.7, 0.5}

\newcommand{\bbdef}[0]{\tcolorbox[colframe=cdef, title=Definition, breakable, enhanced, beforeafter skip=0.5cm]}
\newcommand{\ebdef}{\endtcolorbox}

\newcommand{\bbzum}[0]{\tcolorbox[colframe=cdef, title=Zusammenhang, breakable, enhanced, beforeafter skip=0.5cm]}
\newcommand{\ebzum}{\endtcolorbox}

\newcommand{\bbsatz}[0]{\tcolorbox[colframe=cdef, title=Satz, breakable, enhanced, beforeafter skip=0.5cm]}
\newcommand{\ebsatz}{\endtcolorbox}

\newcommand{\bbgilt}[0]{\tcolorbox[colframe=cdef, title=Es gilt:, breakable, enhanced, beforeafter skip=0.5cm]}
\newcommand{\ebgilt}{\endtcolorbox}

\newcommand{\bbbsp}[0]{\tcolorbox[colframe=cbsp, title=Beispiel, breakable, enhanced, beforeafter skip=0.5cm]}
\newcommand{\ebbsp}{\endtcolorbox}

\newcommand{\bbaufg}[1]{\tcolorbox[colframe=caufg, title={#1}, breakable, enhanced, beforeafter skip=0.5cm]}
\newcommand{\ebaufg}{\endtcolorbox}

\newcommand{\bbmaxima}[0]{\tcolorbox[colframe=cmaxima, title=Maxima, breakable, enhanced, beforeafter skip=0.5cm]}
\newcommand{\ebmaxima}{\endtcolorbox}

\newcommand{\bbfazit}[0]{\tcolorbox[colframe=cnorm, title=Fazit, breakable, enhanced, beforeafter skip=0.5cm]}
\newcommand{\ebfazit}{\endtcolorbox}

\newcommand{\fig}{\begin{figure}}
\newcommand{\efig}{\end{figure}}

\newcommand{\btwocol}{\begin{multicols}{2}}
\newcommand{\etwocol}{\end{multicols}}

\newcommand{\bthreecol}{\begin{multicols}{3}}
\newcommand{\ethreecol}{\end{multicols}}
\ifLuaTeX
  \usepackage{selnolig}  % disable illegal ligatures
\fi
\IfFileExists{bookmark.sty}{\usepackage{bookmark}}{\usepackage{hyperref}}
\IfFileExists{xurl.sty}{\usepackage{xurl}}{} % add URL line breaks if available
\urlstyle{same}
\hypersetup{
  pdftitle={Data Science Mitschrift},
  pdfauthor={Andreas Sünder, Benjamin Kissinger, Yusuf Akalin},
  hidelinks,
  pdfcreator={LaTeX via pandoc}}

\title{Data Science Mitschrift}
\author{Andreas Sünder, Benjamin Kissinger, Yusuf Akalin}
\date{17.01.2022}

\begin{document}
\maketitle

{
\setcounter{tocdepth}{5}
\tableofcontents
}
\graphicspath{ {./Bilder/} }

\newenvironment{conditions}[1][wobei:]
  {#1 \begin{tabular}[t]{>{$}l<{$} @{${} \quad ... \quad {}$} l}}
  {\end{tabular}\\[\belowdisplayskip]}

\clearpage

\hypertarget{einfuxfchrung-in-data-science}{%
\section{Einführung in Data
Science}\label{einfuxfchrung-in-data-science}}

\hypertarget{begriffserkluxe4rungen}{%
\subsection{Begriffserklärungen}\label{begriffserkluxe4rungen}}

\begin{itemize}
\item
  \textbf{Artificial Intelligence}: Teilbereich der Informatik mit
  Automatisierung von Prozessen mit dem Ziel, menschliche
  Intelligenzleistung wie das Lernen oder das Lösen von Problemen
  nachzuahmen.
\item
  \textbf{Machine Learning}: Teilbereich der AI zum Erkennen von Mustern
  und Gesetzmäßigkeiten basierend auf Datenbanken und Algorithmen.
\item
  \textbf{Deep Learning (DL)}: Optimierung künstlicher neuronaler Netze
  mit mehreren Zwischenschichten zwischen Eingabe- und Ausgabeschicht.
\item
  \textbf{Data Science}: ist die Verknüpfung von Statistik und
  Softwareentwicklung Pipelining von Datenbanksystemen und Maschinellem
  Lernen zur Erkennung von Mustern und und Gesetzmäßigkeiten in großen
  Datenengen.
\item
  \textbf{Data Mining (DM):} statistische und ML Algorithmen, die in
  großen Datenbanken Trends und Vernetzungen suchen.
\end{itemize}

\hypertarget{arten-und-ziele-der-datenanalyse}{%
\subsection{Arten und Ziele der
Datenanalyse}\label{arten-und-ziele-der-datenanalyse}}

Die \textbf{expolorative Datenanalyse} beschäftigt sich mit der
Organisation, Zusammenfassung und Visualisierung von Daten. Dazu werden
am häufigsten Grafiken, Tabellen und Schäzer als Hilsmittel verwendet.
Durch eine weitere Analyse und Modellierung der Daten können
\textbf{Modelle gebildet} und \textbf{Hypothesen} getestet werden.
\clearpage

\hypertarget{skalen-und-visualisierung}{%
\section{Skalen und Visualisierung}\label{skalen-und-visualisierung}}

\hypertarget{messung-und-skalierung-von-variablen}{%
\subsection{Messung und Skalierung von
Variablen}\label{messung-und-skalierung-von-variablen}}

Variablen können aufgrund ihrer Eigenschaften in zwei Kategorien geteilt
werden:

\begin{itemize}
\tightlist
\item
  kategoriale Variablen; diskrete Kategorien

  \begin{itemize}
  \tightlist
  \item
    nominal; ohne Ordnung
  \item
    ordinal; geordnet
  \end{itemize}
\item
  metrisch; numerische Zählungen und Messungen

  \begin{itemize}
  \tightlist
  \item
    intervallskaliert; geordnet, Differenzen interpretierbar, aber
    Quotienten NICHT (\(-\infty\), \(\infty\))
  \item
    rational skaliert; geordnet, Differenzen und Quotienten
    interpretierbar; (\(0\), \(\infty\)), absolute Null
  \end{itemize}
\end{itemize}

\hypertarget{huxe4ufigkeiten}{%
\subsection{Häufigkeiten}\label{huxe4ufigkeiten}}

\tcolorbox

Eine \textbf{Häufigkeitsverteilung} (oder \textbf{Häufigkeitstabelle})
gibt die Anzahlen oder die Anteils aller bestimmten
\underline{Kategorien} zugeordneten Beobachtungen wieder.

\vspace{2mm}

Unterschieden werden folgende Unterteilungen:

\begin{itemize}
\tightlist
\item
  absolute Häufigkeiten
\item
  relative Häufigkeiten
\item
  kummulative (absolute bzw. relative) Häufigkeiten
\end{itemize}

\endtcolorbox

Ein Beispiel:

\includegraphics[width=12cm, center]{Fig1}

\hypertarget{bar-charts}{%
\subsubsection{Bar charts}\label{bar-charts}}

Balken- bzw. Säulendiagramme (\textbf{bar charts}) visualisieren
absolute oder relative kumulative und nichtkumulative Häufigkeiten von
Kategorien (empfehlenswert in 90\% aller Fälle) das menschliche Auge und
Gehirn kann kleine Unterschiede bei Längen erkennen und gut
unterscheiden; Flächen sind nur bis zu 20-25 verschiedenen Balken
sinnvolle Informationsträger, dann für die Augen zu verwirrend.

Ein Beispiel:

\begin{Shaded}
\begin{Highlighting}[]
\FunctionTok{barplot}\NormalTok{(VADeaths); }\FunctionTok{barplot}\NormalTok{(VADeaths,}\AttributeTok{beside =} \ConstantTok{TRUE}\NormalTok{)}
\end{Highlighting}
\end{Shaded}

\includegraphics[width=0.5\linewidth]{Mitschrift_files/figure-latex/unnamed-chunk-1-1}
\includegraphics[width=0.5\linewidth]{Mitschrift_files/figure-latex/unnamed-chunk-1-2}

\hypertarget{cleveland-dot-charts}{%
\subsubsection{Cleveland dot charts}\label{cleveland-dot-charts}}

\textbf{Cleveland dot charts} visualisieren absolute oder relative
Häufigkeiten. Das menschliche Auge und Gehirn kann kleine Unterschiede
bei Längen erkennen und gut unterscheiden; wenn viele Häufigkeiten
dargestellt werden müssen (mehr als 25) und die Flächen von
Balkendiagrammen nicht geeignet sind, reduziert man die Information auf
Punkte

Ein Beispiel:

\begin{Shaded}
\begin{Highlighting}[]
\FunctionTok{dotchart}\NormalTok{(VADeaths,}\AttributeTok{main=}\StringTok{"Death Rates in Virginia {-} 1940"}\NormalTok{)}
\end{Highlighting}
\end{Shaded}

\begin{center}\includegraphics{Mitschrift_files/figure-latex/unnamed-chunk-2-1} \end{center}

\hypertarget{pie-charts}{%
\subsubsection{Pie charts}\label{pie-charts}}

Tortendigramme (\textbf{pie charts}) visualisieren ausschließlich
relative Häufigkeiten - absolute Häufigkeiten können damit nicht
sinnvoll dargestellt werden. Relative Häufigkeiten werden auf die
entsprechenden Anteile von 360°umgerechnet und als Kreissegmente
eingezeichnet: das menschliche Auge und Gehirn kann kleine Unterschiede
bei Winkeln aber schlecht erkennen und unterscheiden.

Ein Beispiel:

\begin{Shaded}
\begin{Highlighting}[]
\NormalTok{pie.sales }\OtherTok{\textless{}{-}} \FunctionTok{c}\NormalTok{(}\FloatTok{0.12}\NormalTok{, }\FloatTok{0.3}\NormalTok{, }\FloatTok{0.26}\NormalTok{, }\FloatTok{0.16}\NormalTok{, }\FloatTok{0.04}\NormalTok{, }\FloatTok{0.12}\NormalTok{)}
\FunctionTok{pie}\NormalTok{(pie.sales)}
\end{Highlighting}
\end{Shaded}

\begin{center}\includegraphics{Mitschrift_files/figure-latex/unnamed-chunk-3-1} \end{center}

\hypertarget{charakteristiken-von-metrischen-daten}{%
\subsection{Charakteristiken von metrischen
Daten}\label{charakteristiken-von-metrischen-daten}}

\tcolorbox

\begin{enumerate}
\def\labelenumi{\arabic{enumi}.}
\item
  \textbf{Lage (Lokation)}: beschreibt die Mitte/den Durchschnitt der
  Daten Modalität beschreibt, wieviele Datenzentren es gibt
\item
  \textbf{Streuung (Variation)}: misst, wie stark die Daten um den
  Mittelwert schwanken
\item
  \textbf{Symmetrie/Schiefe}: beschreibt, ob die Daten annähernd
  symmetrisch oder deutlich schief zum oberen oder unteren Rand hin
  verlaufen
\item
  \textbf{Ränder und Ausreißer}: Ränder sind die Werte in der weitesten
  Entfernung vom Zentrum der Daten. Ausreißer sind Beobachtungen, die
  ein anderes Verhalten als die übrigen Daten zeigen
\item
  \textbf{Zeit}: Prozesse, deren Eigenschaften sich abhängig von der
  Zeit ändern
\end{enumerate}

\endtcolorbox

\hypertarget{exkurs-quantile}{%
\subsubsection{Exkurs: Quantile}\label{exkurs-quantile}}

\tcolorbox

Quantile werden aus der geordneten Stichprobe
\(x_{(1)} = \text{min}, \, x_{(2)}, \, ... \, , \, x_{(n-1)}, \, = \text{max}\)
berechnet:

\[\text{Quantil eines Wertes x} \, = \cfrac{\text{Anzahl der Werte} \leq x}{\text{Gesamtazhal der Werte}}\]
\endtcolorbox

Mittels der Funktion \texttt{quantile()} lassen sich die einzelnen
Quantile eines Datensatzes anzeigen:

\begin{Shaded}
\begin{Highlighting}[]
\NormalTok{x }\OtherTok{\textless{}{-}}\NormalTok{ x }\OtherTok{\textless{}{-}} \FunctionTok{c}\NormalTok{(}\DecValTok{5}\NormalTok{, }\DecValTok{3}\NormalTok{, }\DecValTok{8}\NormalTok{, }\DecValTok{2}\NormalTok{, }\DecValTok{6}\NormalTok{, }\DecValTok{1}\NormalTok{, }\DecValTok{9}\NormalTok{, }\DecValTok{4}\NormalTok{)}
\FunctionTok{quantile}\NormalTok{(}\AttributeTok{x =}\NormalTok{ x, }\AttributeTok{probs =} \FunctionTok{c}\NormalTok{(}\FloatTok{0.2}\NormalTok{, }\FloatTok{0.4}\NormalTok{, }\FloatTok{0.5}\NormalTok{, }\FloatTok{0.6}\NormalTok{, }\FloatTok{0.8}\NormalTok{))}
\end{Highlighting}
\end{Shaded}

\begin{verbatim}
## 20% 40% 50% 60% 80% 
## 2.4 3.8 4.5 5.2 7.2
\end{verbatim}

Wichtige Quantile sind:

\tcolorbox

\textbf{Median} = 50\% Quantil.

\vspace{2mm}

Die \underline{Quartile} sind die 25\% und 75\% Qunatile. Die Daten
werden durch die Quartile und den Median ``geviertelt''. Ein robustes
Maß für die Variation ist basiert auf den Quartile, die
Interquartilsdistanz: \[\text{IQD} = X_{0.75} - X_{0.25}\]

\endtcolorbox

\hypertarget{lageschuxe4tzer-lokation}{%
\subsubsection{Lageschätzer (Lokation)}\label{lageschuxe4tzer-lokation}}

\tcolorbox

\begin{center}
$$
\begin{array}{l|c|l}
&\text{Formel}&\text{Code}\\
\hline
\hline
&&\\[-0.25cm]
\text{(Arithmetischer) Mittelwert} & \displaystyle \bar x =  \frac{1}{n}\sum_{i=1}^n x_i & \text{mean(x)}\\
\hline
\text{gewichteter Mittelwert} &  \displaystyle \bar x = \sum_{i=1}^n w_i x_i & \\
& \sum_{i=1}^n{w_i}=1 & \\
\text{getrimmter Mittelwert} &  \displaystyle \bar x =  \frac{1}{n}\sum_{i=q_{trim}}^{q_{1-trim}} x_{(i)}  & mean(x,trim=p)\\
\hline
\text{Geometrischer Mittelwert} &\displaystyle \sqrt[n]{\prod_{i=1}^n x_i} & exp(mean(log(x))) \\

 

\hline
&&\\[-0.25cm]
\text{Harmonischer Mittelwert} & \displaystyle \frac{n}{\sum_{i=1}^n \frac{1}{x_i}} & 1/mean(1/x) \\
\end{array}  
$$
\end{center}

\endtcolorbox

\tcolorbox

\begin{center}
$$
\begin{array}{l|c|l}
&\text{Formel}&\text{Code}\\
\hline
\hline
&&\\[-0.25cm]
\text{Median} & \text{mittlerer Wert} & \\ & \text{der geordneten Daten}&median\\
\hline
\text{Modus} & \text{Wert mit der} & \\ & \text{größten Häufigkeit} & - \\
\hline
&&\\[-0.25cm]
\text{Midrange} & \displaystyle \frac{\max x_i + \min x_i}{2} &-\\
\end{array} 
$$
\end{center}

\endtcolorbox

\tcolorbox

Der \underline{Median} liegt genau in der Mitte der Datenverteilung und
ist robust, sprich, er wird von Ausreißern nicht beeinflusst.

Bei einer

\begin{itemize}
\tightlist
\item
  ungeraden Anzahl von Werten ist der Median genau der Wert in der Mitte
\item
  geraden Anzahl ist der Median der Mittelwert zwischen den beiden
  Werten in der Mitte
\end{itemize}

der Daten. \endtcolorbox

\tcolorbox

Der \textbf{Modus} (\textbf{Modalwert}) ist der Wert mit der höchsten
Wahrscheinlichkeit der zugrundliegenden Verteilung. Bei einer diskreten
oder kategorialen Stichprobe ist es die am häufigsten vorkommende
Kategorie.

\vspace{1mm}

Wichtig ist hierbei zu beachten, dass es auch mehrere \underline{Modi}
geben kann, wodurch zwischen \underline{Unimodalität},
\underline{Bimodalität} und \underline{Multimodalität} unterschieden
werden kann:

\vspace{5mm}

\includegraphics[width=0.33\linewidth]{Mitschrift_files/figure-latex/unnamed-chunk-5-1}
\includegraphics[width=0.33\linewidth]{Mitschrift_files/figure-latex/unnamed-chunk-5-2}
\includegraphics[width=0.33\linewidth]{Mitschrift_files/figure-latex/unnamed-chunk-5-3}
\endtcolorbox

\hypertarget{streuungsschuxe4tzer-variation}{%
\subsubsection{Streuungsschätzer
(Variation)}\label{streuungsschuxe4tzer-variation}}

\includegraphics[width=12cm, center]{Fig4}

\tcolorbox

Die \textbf{Varianz} entspricht der Abweichung der einzelnen Werte von
Mittelwert und Quadrat der Standardabweichung wird sehr wohl von
Ausreißern beeinflusst. \endtcolorbox

\tcolorbox

Die \textbf{Standardabweichung} ist die Streubreite der Werte rund um
den Mittelwert und Quadratwurzelder Varianz). \endtcolorbox

\tcolorbox

Der \textbf{Interquartilsdistanz} (auch IQR genannt) ist als die
Differenz zwischen dem oberen (75\%-) und dem unteren (25\%-) Quartil
definiert. \endtcolorbox

\tcolorbox

Die \textbf{Spannweite} \(x_{max} - x_{min}\) misst den Abstand zwischen
dem größten und kleinsten Messwert und schätzt damit ab, welche
Größenordnung von Werten von den Messungen insgesamt ``über-spannt''
wird. \endtcolorbox

\hypertarget{symmetrieschiefe}{%
\subsubsection{Symmetrie/Schiefe}\label{symmetrieschiefe}}

\tcolorbox

Eine Funktion wird als \textbf{symmetrisch} bezeichnet, wenn sie durch
Spiegelung eines Teilverlaufs an einer Achse oder einem Punkt zustande
kommt. Dabei kann wieder unterschieden werden:

\includegraphics[width=0.5\linewidth]{Mitschrift_files/figure-latex/unnamed-chunk-6-1}
\includegraphics[width=0.5\linewidth]{Mitschrift_files/figure-latex/unnamed-chunk-6-2}
\endtcolorbox

\tcolorbox

Wenn die Verteilung der Daten in einer Richtung \underline{steiler} und
der anderen Richtung \underline{schief auslaufend} verläuft, so spricht
man dann von einer \textbf{schiefen Verteilung}. Je nachdem, in welche
Richtung der lange Rand schief ausläuft, wird diese dann entweder
\underline{rechtschief} oder \underline{linksschief} bezeichnet:

\includegraphics[width=0.5\linewidth]{Mitschrift_files/figure-latex/unnamed-chunk-7-1}
\includegraphics[width=0.5\linewidth]{Mitschrift_files/figure-latex/unnamed-chunk-7-2}
\endtcolorbox

\hypertarget{beispiel-kuxf6rpergruxf6uxdfen}{%
\subsubsection{Beispiel:
Körpergrößen}\label{beispiel-kuxf6rpergruxf6uxdfen}}

Um die einzelnen Maße der Lokation und Streuung noch besser zu
verstehen, soll das Beispiel ``Körpergößen'' näher betrachtet werden.
Definieren wir nun die Werte:

\begin{longtable}[]{@{}ll@{}}
\toprule\noalign{}
\endhead
\bottomrule\noalign{}
\endlastfoot
Körpergrößen & 1.82, 1.75, 1.89, 176, 1.65, 1.71 \\
\end{longtable}

Entsprechender R-Code:

\begin{Shaded}
\begin{Highlighting}[]
\NormalTok{kg}\OtherTok{\textless{}{-}}\FunctionTok{c}\NormalTok{(}\FloatTok{1.82}\NormalTok{, }\FloatTok{1.75}\NormalTok{, }\FloatTok{1.89}\NormalTok{, }\DecValTok{176}\NormalTok{, }\FloatTok{1.65}\NormalTok{, }\FloatTok{1.71}\NormalTok{)}
\end{Highlighting}
\end{Shaded}

Schnell wird klar, dass ein Ausreißer, der Wert \(176\), vorhanden ist.
Nun soll das Verhalten der einzelnen Maße näher betrachtet werden.

\textbf{Mittelwert:}

\begin{Shaded}
\begin{Highlighting}[]
\FunctionTok{mean}\NormalTok{(kg) }\CommentTok{\# 30.80333}
\end{Highlighting}
\end{Shaded}

Man sieht hier deutlich, dass der arithmetische Mittelwert sehr sensitiv
gegenüber Ausreißern ist. Des Weiteren wird dieser auch durch Asymmetrie
und Rändern zerstört bzw. verzerrt. Schaut man sich das Bild des Medians
an, so sieht man hier einen klaren Unterschied:

\textbf{Median:}

\begin{Shaded}
\begin{Highlighting}[]
\FunctionTok{median}\NormalTok{(kg) }\CommentTok{\# 1.785}
\end{Highlighting}
\end{Shaded}

Der eine Ausreißer tut dem Median offenbar nicht wirklich weh, er ist
also \underline{robust}. Selbst ein Anteil von 50\% an fehlerhaften
Werten beeinflussen den Median nur geringfügig.

\textbf{Varianz:}

In diesem Fall besitzt die Varianz einen sehr hohen Wert, da der eine
Ausreißer (gemäß der zu anwendenden Formel) insgesamt sechs Mal
quadriert wird:

\begin{Shaded}
\begin{Highlighting}[]
\FunctionTok{var}\NormalTok{(kg) }\CommentTok{\# 5059.704}
\end{Highlighting}
\end{Shaded}

\textbf{Standardabweichung:}

\ldots{} dementsprechend ist die Standardabweichung auch nicht robust:

\begin{Shaded}
\begin{Highlighting}[]
\FunctionTok{sd}\NormalTok{(kg) }\CommentTok{\# 71.1316}
\end{Highlighting}
\end{Shaded}

\textbf{Interquartilsdistanz:}

Durch seine Eigenschaften ist die IQR nicht robust:

\begin{Shaded}
\begin{Highlighting}[]
\FunctionTok{IQR}\NormalTok{(kg) }\CommentTok{\# 0.1525}
\end{Highlighting}
\end{Shaded}

\hypertarget{darstellung-von-metrischen-daten}{%
\subsection{Darstellung von metrischen
Daten}\label{darstellung-von-metrischen-daten}}

\hypertarget{boxplot}{%
\subsubsection{Boxplot}\label{boxplot}}

\tcolorbox

Der \textbf{Boxplot} stellt den Median als Mitte der Box, die Quartile
als die beiden Enden der Box dar, was die mittleren 50 \% der Daten klar
kennzeichnet. Darüber hinaus werden durch die Whiskers, die maximal das
1,5-fache des Quartilsabstands umfassen, die zentralen 95 \% der Daten
abgesteckt, wenn die Daten normalverteilt wären. Daher werden die Werte
außerhalb der Whiskers als potentielle Ausreißer bezeichnet. Wenn die
Verteilung der Daten aber inherent schief oder von schweren Rändern
geprägt ist, ist diese Ausreißereinteilung jedenfalls falsch.

Ein Beispiel:

\begin{Shaded}
\begin{Highlighting}[]
\FunctionTok{boxplot}\NormalTok{(airquality}\SpecialCharTok{$}\NormalTok{Ozone,}
  \AttributeTok{main =} \StringTok{"Mean ozone in parts per billion at Roosevelt Island"}\NormalTok{,}
  \AttributeTok{xlab =} \StringTok{"Parts Per Billion"}\NormalTok{,}
  \AttributeTok{ylab =} \StringTok{"Ozone"}\NormalTok{,}
  \AttributeTok{horizontal =} \ConstantTok{TRUE}\NormalTok{,}
  \AttributeTok{cex.main =} \FloatTok{1.1}
\NormalTok{)}
\end{Highlighting}
\end{Shaded}

\begin{center}\includegraphics{Mitschrift_files/figure-latex/unnamed-chunk-14-1} \end{center}
\endtcolorbox

\hypertarget{histogramme-und-stetige-dichteschuxe4tzer}{%
\subsubsection{Histogramme und stetige
Dichteschätzer}\label{histogramme-und-stetige-dichteschuxe4tzer}}

\tcolorbox

Ein \textbf{Histogramm} ist ein Graph, der sich aus Balken
zusammensetzt, deren Höhe die Anzahl der Anteil der Daten innerhalb von
Teilintervallen ist.

\vspace{2mm}

Ein \textbf{Kerndichteschätzer} ist ein Graph, der die Verteilung der
Daten durch eine Approximation mithilfe von Näherungsfunktionen (Kernen)
stetig verlaufend wiedergibt.

Ein Beispiel:

\begin{Shaded}
\begin{Highlighting}[]
\NormalTok{daten }\OtherTok{\textless{}{-}}\NormalTok{ airquality}\SpecialCharTok{$}\NormalTok{Temp}
\FunctionTok{hist}\NormalTok{(daten, }\AttributeTok{main=}\StringTok{"Histogramm"}\NormalTok{, }\AttributeTok{xlab=}\StringTok{"Beobachtungen"}\NormalTok{, }\AttributeTok{ylab=}\StringTok{"Häufigkeit"}\NormalTok{, }\AttributeTok{freq =} \ConstantTok{FALSE}\NormalTok{);}
\FunctionTok{lines}\NormalTok{(}\FunctionTok{density}\NormalTok{(daten))}
\end{Highlighting}
\end{Shaded}

\begin{center}\includegraphics[width=0.6\linewidth]{Mitschrift_files/figure-latex/unnamed-chunk-15-1} \end{center}

Mit Histogrammen können Eigenschaften wie Modalität oder Symmetrie der
Daten klar dargestellt werden. \endtcolorbox

\hypertarget{vergleich-histogramm-und-boxplot}{%
\subsubsection{Vergleich Histogramm und
Boxplot}\label{vergleich-histogramm-und-boxplot}}

Mit dem eigenen Datensatz \texttt{oscars} soll hier gezeigt werden, dass
es durchaus sinnvoll ist, ein Histogramm und ein Boxplot kombiniert
darzustellen:

\begin{center}\includegraphics[width=0.7\linewidth]{Mitschrift_files/figure-latex/unnamed-chunk-16-1} \end{center}

\begin{center}\includegraphics[width=0.7\linewidth]{Mitschrift_files/figure-latex/unnamed-chunk-17-1} \end{center}

\hypertarget{quantil-quantil-plot-qq-plot}{%
\subsubsection{Quantil-Quantil-Plot
(QQ-Plot)}\label{quantil-quantil-plot-qq-plot}}

\tcolorbox

Ein \textbf{Quantil-Quantil-Plot (QQ-Plot)} ist ein Graph, der die
Verteilung der Daten einer Stichprobe mit der Verteilung der Daten einer
anderen Stichprobe oder einer theoretischen Verteilung der Datenwerte
(z. B. Standardnormalverteilung) vergleicht. \endtcolorbox

Ein Beispiel:

\begin{Shaded}
\begin{Highlighting}[]
\FunctionTok{qqnorm}\NormalTok{(normaleDaten); }\FunctionTok{qqline}\NormalTok{(normaleDaten,}\AttributeTok{col=}\DecValTok{2}\NormalTok{)}
\FunctionTok{qqnorm}\NormalTok{(gammaDaten); }\FunctionTok{qqline}\NormalTok{(gammaDaten,}\AttributeTok{col=}\DecValTok{2}\NormalTok{)}
\end{Highlighting}
\end{Shaded}

\includegraphics[width=0.5\linewidth]{Mitschrift_files/figure-latex/unnamed-chunk-19-1}
\includegraphics[width=0.5\linewidth]{Mitschrift_files/figure-latex/unnamed-chunk-19-2}

\clearpage

\# Abhängigkeit von 2 oder mehr Variablen

\hypertarget{zusammenhuxe4nge-zwischen-2-metrischen-variablen}{%
\subsection{Zusammenhänge zwischen 2 metrischen
Variablen}\label{zusammenhuxe4nge-zwischen-2-metrischen-variablen}}

\tcolorbox

Wenn man von Regression spricht, versucht man zu analysieren, eine
beobachtete \textbf{abhängige} Variable durch eine (oder mehrere, siehe
unten) \textbf{unabhängige} Variablen zu erklären. Bei linearer
Regression geht man hierbei von einem linearen Modell aus:

\[y = \alpha \; + \beta \; \cdot \; x\]

\vspace{3mm}

\includegraphics[width=10cm, center]{Fig7} \vspace{3mm}

Allgemein legt man einer solchen Anpassung, welche Modellierung mittels
linearer Regression genannt wird, eine \textbf{Regressionsgleichung}
zugrunde:

\[y_i = \alpha \; + \; \beta \; \cdot \; x_i \; + \; \epsilon_i\]

\bgroup wobei: \begin{tabular}[t]{>{$}l<{$} @{${} \quad ... \quad {}$} l}
\alpha  &   Achsenabschnitt (y-Wert, wenn x den Wert 0 annimmt) \\
\beta   &   Steigung der Gerade (y steigt um $\beta$ Einheiten, wenn x um 1 vergrößert wird) \\
\epsilon_i & Residuenfehler der Punkte (vertikale Entferung der Gerade vom tatsächlichen Punkt)
\end{tabular}\\[\belowdisplayskip]\egroup

Diese Parameter haben auch eine mathematische Bedeutung:

\vspace{2mm}

\bgroup wobei: \begin{tabular}[t]{>{$}l<{$} @{${} \quad ... \quad {}$} l}
\alpha  &   Wert auf der y-Achse, an dem die Gerade die y-Achse schneidet, bzw. Startwert für x=0 \\
\beta > 0   &   ist die Gerade steigend \\
\beta < 0   &   ist die Gerade fallend
\end{tabular}\\[\belowdisplayskip]\egroup

\endtcolorbox

\tcolorbox

\textbf{Korrelation}

Als standardisiertes Maß fürr die lineare Abhängigkeit wird der
\textbf{Pearson Korrelationskoeffizient} definiert:

\[r = r(X,Y) = \cfrac{\widehat{cov}(X, Y)}{s(X) \, \cdot \, s(Y)}\]

Bei diesem Koeffizient wird die Achsenskalierung herausgerechnet und
dadurch nimmt er ausschließlich Wert zwischen -1 und 1 an:

\includegraphics[width=0.33\linewidth]{Mitschrift_files/figure-latex/unnamed-chunk-20-1}
\includegraphics[width=0.33\linewidth]{Mitschrift_files/figure-latex/unnamed-chunk-20-2}
\includegraphics[width=0.33\linewidth]{Mitschrift_files/figure-latex/unnamed-chunk-20-3}

\bgroup wobei: \begin{tabular}[t]{>{$}l<{$} @{${} \quad ... \quad {}$} l}
negatives Vorzeichen & fallender Zusammenhang \\
positives Vorzeichen & steigender Zusammenhang
\end{tabular}\\[\belowdisplayskip]\egroup

Faustregel zur Interpreation des Korrelationskoeffizienten:

\vspace{2mm}

\bgroup wobei: \begin{tabular}[t]{>{$}l<{$} @{${} \quad ... \quad {}$} l}
r_{(s)} = 0   &     keine Korrelation \\
0 \le |r_{(s)}| \leq 0.5 & schwache Korrelation \\
0.5 \le |r_{(s)}| \leq 0.75 & mittlere Korrelation \\
0.75 \le |r_{(s)}| \le 1 & starke Korrelation \\
r_{(s)} = 1   &     vollständige Korrelation
\end{tabular}\\[\belowdisplayskip]\egroup

\endtcolorbox

\clearpage

\hypertarget{multiple-regresion}{%
\subsection{Multiple Regresion}\label{multiple-regresion}}

\tcolorbox

Wenn wir anstatt nur einer unabhängigen erklärenden Variablen mehrere
benutzen, gehen wir von simplen linearen Regressionsmodell zum multiplen
linearen Regressionsmodell über:

\[y_i = \alpha \, + \, \beta_1x_{1,i} \, + \, \beta_2x_{2,i} \, + \; ... \, + \beta_kx_{k,i} \, + \, \epsilon_i\]

\endtcolorbox

\tcolorbox

\textbf{Annahmen und Voraussetzungen für multiple Regression:}

(A1) Das Modell hat keinen systematischen Fehler.

(A2) Die Fehlervarianz ist fur alle Beobachtungen gleich groß
(homoskedastisch).

(A3) Die Komponenten des Fehlerterms sind nicht korreliert.

(A4) Der Modellfehler sei normalverteilt.

(A5) Es gibt keine linearen Abhängigkeiten zwischen den Regressoren.
\endtcolorbox

\hypertarget{beispiel-mit-dem-datensatz-state.x77}{%
\subsubsection{\texorpdfstring{Beispiel mit dem Datensatz
\texttt{state.x77}:}{Beispiel mit dem Datensatz state.x77:}}\label{beispiel-mit-dem-datensatz-state.x77}}

Die ersten vier Annahmen können mit einem \texttt{pairs}-Plot analysiert
werden:

\begin{Shaded}
\begin{Highlighting}[]
\FunctionTok{pairs}\NormalTok{(state.x77[,}\DecValTok{1}\SpecialCharTok{:}\DecValTok{5}\NormalTok{], }\AttributeTok{lower.panel =}\NormalTok{ panel.smooth, }\AttributeTok{upper.panel =}\NormalTok{ panel.cor,}
\AttributeTok{diag.panel =}\NormalTok{ panel.hist, }\AttributeTok{las=}\DecValTok{1}\NormalTok{)}
\end{Highlighting}
\end{Shaded}

\includegraphics{Mitschrift_files/figure-latex/unnamed-chunk-21-1.pdf}

Man sieht hier deutlich, dass Life Expectancy und Illiteracy linear
zusammenhängen, mit einem Korrelationskoeffizienten von 0.59. Bleiben
beide Variablen im Modell drinnen, so würden diese unser Modell
vollkommen zerstören. Daher muss eines von beiden raus.

Murder und Life Expectancy bzw. Murder und Illiteracy dürfen
korrelieren, da Murder die abhängige Variable darstellt.

Entfernen wir nun Life Expectancy, schaut das so aus:

\begin{Shaded}
\begin{Highlighting}[]
\FunctionTok{pairs}\NormalTok{(state.x77[, }\FunctionTok{c}\NormalTok{(}\DecValTok{1}\NormalTok{, }\DecValTok{2}\NormalTok{, }\DecValTok{3}\NormalTok{, }\DecValTok{5}\NormalTok{)], }\AttributeTok{lower.panel =}\NormalTok{ panel.smooth, }\AttributeTok{upper.panel =}\NormalTok{ panel.cor,}
\AttributeTok{diag.panel =}\NormalTok{ panel.hist, }\AttributeTok{las=}\DecValTok{1}\NormalTok{)}
\end{Highlighting}
\end{Shaded}

\includegraphics{Mitschrift_files/figure-latex/unnamed-chunk-22-1.pdf}

\clearpage

\begin{Shaded}
\begin{Highlighting}[]
\FunctionTok{summary}\NormalTok{(}\FunctionTok{lm}\NormalTok{(}\AttributeTok{formula =}\NormalTok{ Murder }\SpecialCharTok{\textasciitilde{}}\NormalTok{ .  }\SpecialCharTok{{-}} \StringTok{\textasciigrave{}}\AttributeTok{Life Exp}\StringTok{\textasciigrave{}} \SpecialCharTok{{-}} \StringTok{\textasciigrave{}}\AttributeTok{HS Grad}\StringTok{\textasciigrave{}} \SpecialCharTok{{-}}\NormalTok{ Frost }\SpecialCharTok{{-}}\NormalTok{ Area,}
           \AttributeTok{data =} \FunctionTok{as.data.frame}\NormalTok{(state.x77)))}
\end{Highlighting}
\end{Shaded}

\begin{verbatim}
## 
## Call:
## lm(formula = Murder ~ . - `Life Exp` - `HS Grad` - Frost - Area, 
##     data = as.data.frame(state.x77))
## 
## Residuals:
##     Min      1Q  Median      3Q     Max 
## -4.7846 -1.6768 -0.0839  1.4783  7.6417 
## 
## Coefficients:
##              Estimate Std. Error t value Pr(>|t|)    
## (Intercept) 1.3402721  3.3694210   0.398   0.6926    
## Population  0.0002219  0.0000842   2.635   0.0114 *  
## Income      0.0000644  0.0006762   0.095   0.9245    
## Illiteracy  4.1109188  0.6706786   6.129 1.85e-07 ***
## ---
## Signif. codes:  0 '***' 0.001 '**' 0.01 '*' 0.05 '.' 0.1 ' ' 1
## 
## Residual standard error: 2.507 on 46 degrees of freedom
## Multiple R-squared:  0.5669, Adjusted R-squared:  0.5387 
## F-statistic: 20.07 on 3 and 46 DF,  p-value: 1.84e-08
\end{verbatim}

Dieses Modell ist für uns sehr schlecht, da bei drei Variablen die
Irrtumswahrscheinlichkeit sehr hoch ist (bis 92 \%!). Außerdem ist bei
der Variable Income der Standard-Error größer als das Estimate selbst
ist. Zusammen mit der extrem hohen Irrtumswahrscheinlichkeit kann diese
Variable sehr schnell negativ werden.

\tcolorbox

\textbf{Fazit:} Dieses Modell ist für uns nicht geeignet! \endtcolorbox

Schmeißen wir nun Illiteracy statt Life Expectancy raus:

\begin{Shaded}
\begin{Highlighting}[]
\FunctionTok{pairs}\NormalTok{(state.x77[, }\FunctionTok{c}\NormalTok{(}\DecValTok{1}\NormalTok{, }\DecValTok{2}\NormalTok{, }\DecValTok{4}\NormalTok{, }\DecValTok{5}\NormalTok{)], }\AttributeTok{lower.panel =}\NormalTok{ panel.smooth, }\AttributeTok{upper.panel =}\NormalTok{ panel.cor,}
\AttributeTok{diag.panel =}\NormalTok{ panel.hist, }\AttributeTok{las=}\DecValTok{1}\NormalTok{)}
\end{Highlighting}
\end{Shaded}

\includegraphics{Mitschrift_files/figure-latex/unnamed-chunk-24-1.pdf}

\clearpage

\begin{Shaded}
\begin{Highlighting}[]
\FunctionTok{summary}\NormalTok{(}\FunctionTok{lm}\NormalTok{(}\AttributeTok{formula =}\NormalTok{ Murder }\SpecialCharTok{\textasciitilde{}}\NormalTok{ .  }\SpecialCharTok{{-}}\NormalTok{ Illiteracy }\SpecialCharTok{{-}} \StringTok{\textasciigrave{}}\AttributeTok{HS Grad}\StringTok{\textasciigrave{}} \SpecialCharTok{{-}}\NormalTok{ Frost }\SpecialCharTok{{-}}\NormalTok{ Area,}
           \AttributeTok{data =} \FunctionTok{as.data.frame}\NormalTok{(state.x77)))}
\end{Highlighting}
\end{Shaded}

\begin{verbatim}
## 
## Call:
## lm(formula = Murder ~ . - Illiteracy - `HS Grad` - Frost - Area, 
##     data = as.data.frame(state.x77))
## 
## Residuals:
##     Min      1Q  Median      3Q     Max 
## -5.0604 -1.2165 -0.1001  1.4489  5.3766 
## 
## Coefficients:
##               Estimate Std. Error t value Pr(>|t|)    
## (Intercept)  1.530e+02  1.638e+01   9.340 3.41e-12 ***
## Population   2.487e-04  6.955e-05   3.576 0.000834 ***
## Income      -2.309e-04  5.362e-04  -0.431 0.668697    
## `Life Exp`  -2.055e+00  2.406e-01  -8.541 4.77e-11 ***
## ---
## Signif. codes:  0 '***' 0.001 '**' 0.01 '*' 0.05 '.' 0.1 ' ' 1
## 
## Residual standard error: 2.102 on 46 degrees of freedom
## Multiple R-squared:  0.6957, Adjusted R-squared:  0.6759 
## F-statistic: 35.06 on 3 and 46 DF,  p-value: 6.03e-12
\end{verbatim}

Dieses Modell ist nun viel besser als vorher (mit einem R-squared von
0.68 statt 0.54 vorher). Jedoch passt die Variable Income nach wie vor
nicht. Mit einer Irrtumswahrscheinlichkeit von 68\% ist diese nicht
akzeptabel. Diese gehört also auch weg:

\begin{Shaded}
\begin{Highlighting}[]
\FunctionTok{summary}\NormalTok{(}\FunctionTok{lm}\NormalTok{(}\AttributeTok{formula =}\NormalTok{ Murder }\SpecialCharTok{\textasciitilde{}}\NormalTok{ . }\SpecialCharTok{{-}}\NormalTok{ Income}\SpecialCharTok{{-}}\NormalTok{ Illiteracy }\SpecialCharTok{{-}} \StringTok{\textasciigrave{}}\AttributeTok{HS Grad}\StringTok{\textasciigrave{}} \SpecialCharTok{{-}}\NormalTok{ Frost }\SpecialCharTok{{-}}\NormalTok{ Area,}
           \AttributeTok{data =} \FunctionTok{as.data.frame}\NormalTok{(state.x77)))}
\end{Highlighting}
\end{Shaded}

\begin{verbatim}
## 
## Call:
## lm(formula = Murder ~ . - Income - Illiteracy - `HS Grad` - Frost - 
##     Area, data = as.data.frame(state.x77))
## 
## Residuals:
##     Min      1Q  Median      3Q     Max 
## -4.9310 -1.2066 -0.1546  1.4868  5.3323 
## 
## Coefficients:
##               Estimate Std. Error t value Pr(>|t|)    
## (Intercept)  1.547e+02  1.578e+01   9.805 6.02e-13 ***
## Population   2.413e-04  6.682e-05   3.612 0.000737 ***
## `Life Exp`  -2.093e+00  2.222e-01  -9.417 2.15e-12 ***
## ---
## Signif. codes:  0 '***' 0.001 '**' 0.01 '*' 0.05 '.' 0.1 ' ' 1
## 
## Residual standard error: 2.083 on 47 degrees of freedom
## Multiple R-squared:  0.6945, Adjusted R-squared:  0.6815 
## F-statistic: 53.42 on 2 and 47 DF,  p-value: 7.899e-13
\end{verbatim}

Dieses Modell ist nun das beste. Die Modellgleichung ist nun folgende:

\[
\text{Murder} = 154.7 + 2.413 \cdot 10^{-4} \cdot \text{Population} - 2.093 \cdot \text{Life Exp}
\]

\begin{Shaded}
\begin{Highlighting}[]
\FunctionTok{par}\NormalTok{(}\AttributeTok{mfrow =} \FunctionTok{c}\NormalTok{(}\DecValTok{2}\NormalTok{,}\DecValTok{2}\NormalTok{))}
\FunctionTok{plot}\NormalTok{(}\FunctionTok{lm}\NormalTok{(}\AttributeTok{formula =}\NormalTok{ Murder }\SpecialCharTok{\textasciitilde{}}\NormalTok{ . }\SpecialCharTok{{-}}\NormalTok{ Income}\SpecialCharTok{{-}}\NormalTok{ Illiteracy }\SpecialCharTok{{-}} \StringTok{\textasciigrave{}}\AttributeTok{HS Grad}\StringTok{\textasciigrave{}} \SpecialCharTok{{-}}\NormalTok{ Frost }\SpecialCharTok{{-}}\NormalTok{ Area,}
           \AttributeTok{data =} \FunctionTok{as.data.frame}\NormalTok{(state.x77)))}
\end{Highlighting}
\end{Shaded}

\begin{center}\includegraphics{Mitschrift_files/figure-latex/unnamed-chunk-27-1} \end{center}

\clearpage

\hypertarget{varianzanalyse}{%
\subsection{Varianzanalyse}\label{varianzanalyse}}

\hypertarget{einfache-varianzanalysen-anova}{%
\subsubsection{Einfache Varianzanalysen
(ANOVA)}\label{einfache-varianzanalysen-anova}}

Derweil waren y und x werte numerisch ==\textgreater{} man sagt
Zahlenwerte vorher. Wenn wir als y numerische Variable haben, aber als x
kategorische Variablen ==\textgreater{} ANOVA

\tcolorbox

\textbf{Beispiel:}

Acker testen ==\textgreater{} in mehrere Teile teilen und jeder Teil hat
jeweils unterschiedliche Bedinungen und wird unterschiedlich behandelt
(anderer Dünger). Danach vergleichen, wie gut Ertrag war.

\[ y = \alpha + \beta \cdot x \]

\bgroup wobei: \begin{tabular}[t]{>{$}l<{$} @{${} \quad ... \quad {}$} l}
x & kategroische Variable die Dünger oder Natur kodiert \\
y & Ertrag \\
\alpha & Mittlerer Ertrag aller Ackerflächen \\
\beta & Steigung
\end{tabular}\\[\belowdisplayskip]\egroup

x kann 0 oder 1 annehmen ==\textgreater{} natur oder dünger
==\textgreater{} bestimmte Abweichung zum Referenzwert alpha (oben oder
unten) ==\textgreater{} im Mittel müssel sie sich ausgleichen. Jetzt
müsste man sich fragen, ab was für einen Wert es sich lohnt, in Dünger,
bzw. Werbung aber dafür Naturboden zu investieren.

\textbf{Im Allgemeinen gilt:}

\[ y_{ij} = Mü + \alpha_i + e_{ij} \]

\bgroup wobei: \begin{tabular}[t]{>{$}l<{$} @{${} \quad ... \quad {}$} l}
Mü & Gesamtmittelwert \\
\alpha_i & Abstand der Gruppenmittelwerte von µ \\
e_{ij} & Fehler=Residuen
\end{tabular}\\[\belowdisplayskip]\egroup 
\endtcolorbox

ANOVA kann auch als Hypthesentests verwendet werden ==\textgreater{} Ist
in allen Gruppen der Mittelwert gleich oder nicht? Anderes Beispiel:
Vergleich von Körpergrößen von Jahrgängen ==\textgreater{} sollte
steigen:

\begin{itemize}
\tightlist
\item
  H0: Im Mittel die selbe Körpergröße
\item
  H1: Wenigstens in einem Jahrgang (=Kategorie) ist der Mittelwert
  unterschiedlich.
\end{itemize}

Das ist die allgemeine Methode, die Unterschiede der mittleren Werte von
bekannten Kategorien zu ermitteln.

\tcolorbox

\textbf{Voraussetzungen für ANOVA:}

(A1) die Daten sind annähernd symmetrisch

(A2) die Daten haben nur einen Modus (unimodal)

(A3) die Daten enthalten keine Ausreißer \endtcolorbox

Ab Seite 56 im Skript \textbf{statische Modelle}: Beispiele und Plots.
Wenn oberes und unteres klaren Abstand haben ==\textgreater{} klare
Aussage mit ANOVA ==\textgreater{} mind. 2 Kategorien unterscheiden
sich! Im Optimalfall keine Überschneidungen ==\textgreater{} klare
Aussage!

\includegraphics[width=15cm, center]{Fig8}

Wenn wirs mit freien Auge nicht erkennen können: ANOVA als
Hypthesentests sinnvoll. Wie funktionieren diese Tests?

\tcolorbox

\textbf{Hypothesentests}

\includegraphics[width=10cm, center]{Fig9}

\bgroup wobei: \begin{tabular}[t]{>{$}l<{$} @{${} \quad ... \quad {}$} l}
RSS1 & Schwankung in den Stichproben \\
RSS0 & Schwankung zwischen den Stichproben
\end{tabular}\\[\belowdisplayskip]\egroup

Zu RSS1: Diese Striche sind ja auch nur mehrere Punkte (Messwerte) an
einer x-Koordinate, Strich ist der Mittelwert \emph{in} von diesen
Messpunkten ==\textgreater{} Messgenauigkeit von Messungen

Zu RSS0: Mittelwert von beiden Messpunkten

Optimallfall: RSS0 groß und RSS1 klein. \endtcolorbox

Die Varianzen werden verglichen mit:

\[ RSS0/RSS1 \]

Um gegen Ausreißer vorzugehen, kann man mit Rängen arbeiten
==\textgreater{} Messwerte durchnummerieren und Median nehmen
==\textgreater{} Median robust gegen Ausreißer.

\hypertarget{zweiweg-varianzanalyse-anova}{%
\subsubsection{Zweiweg Varianzanalyse
(ANOVA)}\label{zweiweg-varianzanalyse-anova}}

Wir fügen eine kategoriale Variable hinzu, fragen aber immer noch nach
demselben y-Wert:

\includegraphics[width=10cm, center]{Fig10}

wobei:

\begin{itemize}
\tightlist
\item
  Mü=y=Gesamtmittelwert
\item
  alpha\_i=Abstand der Gruppenmittelwerte von x1-Gruppe i vom
  Gesamtmittelwert
\item
  Beta\_j=Abstand der Gruppenmittelwerte von x2-Gruppe j vom
  Gruppenmittelwert alpha\_i
\end{itemize}

{[}Grafik: Zweiweg-Varianzanalyse{]}

\textbf{Was kann nun passieren?}

\begin{itemize}
\item
  y\textasciitilde1: Alles hat den selben Mittelwert - Nur der
  Mittelwert der gesamten Daten wird als Mittelwert in allen
  Teilkategorien angenommen
\item
  y \textasciitilde{} X1: Nur X1 ist relevant und führt zu einer
  Aufteilung der Mittelwerte
\item
  y \textasciitilde{} X2: Nur X2 ist relevant und führt zu einer
  Aufteilung der Mittelwerte
\item
  X1 + X2: X1 und X2 führen zu einer Aufteilung der Mittelwerte (Echtes
  ANOVA-Modell)
\item
  X1 * X2: X1 und X2 führen zu einer Aufteilung der Mittelwerte in den
  unterschiedlichen Teilkategorien und zusätzlich addieren sich die
  Effekte nicht (Z.B. Säure Boden und Säure Dünger für Pflanze - gut,
  aber basischer Boden und basischer Dünger für Pflanze - schlecht).
\end{itemize}

==\textgreater{}

\begin{itemize}
\tightlist
\item
  links: additiv
\item
  rechts: interaktiv
\end{itemize}

Ab Seite 66 ==\textgreater{} Beispiele

Nicht mehr parallel ==\textgreater{} interaktiv

Modellselektion nach Komplexität:

\begin{enumerate}
\def\labelenumi{\arabic{enumi}.}
\tightlist
\item
  Y \textasciitilde{} X1*X2
\item
  Y \textasciitilde{} X1+X2
\item
  Y \textasciitilde{} X1 und Y \textasciitilde{} X2
\item
  Y \textasciitilde{} 1
\end{enumerate}

\textbf{Interactionplot in R:}

\begin{Shaded}
\begin{Highlighting}[]
\FunctionTok{interaction.plot}\NormalTok{(x.factor}\SpecialCharTok{*}\NormalTok{(x1)}\SpecialCharTok{*} \ErrorTok{=}\NormalTok{ ..., trace.factor}\SpecialCharTok{*}\NormalTok{(x2)}\SpecialCharTok{*} \ErrorTok{=}\NormalTok{ ..., response}\SpecialCharTok{*}\NormalTok{(y)}\SpecialCharTok{*} \ErrorTok{=}\NormalTok{ ...)}
\end{Highlighting}
\end{Shaded}

\textbf{ANOVA models in R:}

\includegraphics[width=10cm, center]{Fig11}

\clearpage

\hypertarget{klassifikation}{%
\subsection{Klassifikation}\label{klassifikation}}

Bei der Regression --\textgreater{} numerische Variablen Bei der
Klassifikation --\textgreater{} kategoriale Variablen

D.h. bei der Klassifikation werden numerischen Werten (x-Achse)
Kategorien zugewiesen (y-Achse).

\hypertarget{binuxe4re-klassifikation}{%
\subsection{binäre Klassifikation}\label{binuxe4re-klassifikation}}

\hypertarget{binomial}{%
\subsubsection{binomial:}\label{binomial}}

2 mögliche Ausgänge, die zufallsbehaftet sind (0 und 1, geschafft und
nicht geschafft, \ldots). Die Wahrscheinlichkeit bleibt dabei gleich
(Beispiel mit 6 Stiften: bei jedem zufälligen Ziehen eines Stiftes muss
die Wahrscheinlichkeit 1/6 sein. Eine Wahrscheinlichkeitsänderung, z.B.
durch Ablage eines Stiftes, darf nicht stattfinden!)\\
Es wird mit einer vorgegebenen Anzahl an Experimenten, die unabhängig
voneinander sind, gearbeitet. (Ist zum Beispiel nicht der Fall bei
Infektionskrankheiten.)

\hypertarget{logistische-regression}{%
\subsubsection{logistische Regression}\label{logistische-regression}}

\includegraphics[width=12cm, center]{Fig5}

Wird auch als verallgemeinerte Regression bezeichnet (engl.: Generalized
Linear Model, R: glm(y\textasciitilde x, data, link =
binomial(``logit''))).

\clearpage

\hypertarget{wahrscheinlichkeit-chancen}{%
\section{Wahrscheinlichkeit \&
Chancen}\label{wahrscheinlichkeit-chancen}}

\begin{longtable}[]{@{}
  >{\raggedright\arraybackslash}p{(\columnwidth - 2\tabcolsep) * \real{0.2903}}
  >{\raggedright\arraybackslash}p{(\columnwidth - 2\tabcolsep) * \real{0.7097}}@{}}
\toprule\noalign{}
\begin{minipage}[b]{\linewidth}\raggedright
Wahrscheinlichkeit
\end{minipage} & \begin{minipage}[b]{\linewidth}\raggedright
Chancen
\end{minipage} \\
\midrule\noalign{}
\endhead
\bottomrule\noalign{}
\endlastfoot
90\%, 5\%, 0.75, \ldots{} & 3:5 (3/8 : 5/8), 1:3 (1/4 : 3/4),
\ldots{} \\
\(0 \leq |x| \leq 1\) & Wahrscheinlichkeit : Gegenwahrscheinlichkeit \\
\end{longtable}

\hypertarget{chancenverhuxe4ltnis}{%
\subsection{Chancenverhältnis}\label{chancenverhuxe4ltnis}}

engl. Odds Ratio, OR

\[
\frac{\frac{3}{5}}{\frac{1}{3}} = \frac{9}{5} = 1.8
\]

Die Chance 3:5 ist 1.8 mal so groß wie die Chance 1:3.

\[
logit(y) = \color{red} \alpha + \color{blue} \beta_1 \cdot x_1 + \color{blue} \beta_2 \cdot x_2 + ... + \epsilon_i
\] {\(\beta\)} : log-Odds-Ratio (e{\(\beta\)1})\\
{\(\alpha\)} : Chance 1 vs 0 (e{\(\alpha\)})

OR \textgreater{} 1 \ldots{} erhöhtes Risiko\\
OR = 1 \ldots{} gleichbleibendes Risiko\\
OR \textless{} 1 \ldots{} geringeres Risiko\\
\clearpage

\hypertarget{fehlerklassifikation}{%
\section{Fehlerklassifikation}\label{fehlerklassifikation}}

Vergleich von Realität und Vorhersage des Modells:

Falschpositiv: ``Signifikanz'', kann reguliert werden. Falschnegativ:
``Schärfe'' (engl. ``Power''), kann weder reguliert noch exakt
vorhergesagt werden.

\hypertarget{receiver-operating-curve-roc}{%
\subsection{Receiver Operating Curve
(ROC)}\label{receiver-operating-curve-roc}}

Vergleicht TP mit FN\\
Alles, was unter der Referenzlinie (Diagonale) ist, ist unbrauchbar, da
das Modell öfter falsche Ergebnisse liefert.

\includegraphics[width=12cm, center]{Fig6}

\hypertarget{auc-area-under-the-curve}{%
\subsubsection{AUC (Area under the
curve)}\label{auc-area-under-the-curve}}

Damit ein Algorithmus sinvoll ist, muss seine AUC \textgreater{} 0.5
sein.

D.h. Die rote Kurve im obigen Bild ist ein besseres Modell als die
grüne, da die Fläche unter der Kurve größer ist. Im Idealfall beträgt
die Fläche unter der Kurve 1.\clearpage

\hypertarget{einfuxfchrung-in-machine-learning}{%
\section{Einführung in Machine
Learning}\label{einfuxfchrung-in-machine-learning}}

\hypertarget{grundlagen}{%
\subsection{Grundlagen}\label{grundlagen}}

Unterbereich von Künstlicher Intelligenz. Hauptthemen:

\begin{itemize}
\tightlist
\item
  Algorithmen\\
\item
  Wie lernt eine Maschine?\\
\item
  Wie prüft man nach Qualität?\\
\item
  Was ist gut? Was ist das Ziel eines Algorithmus?
\end{itemize}

Im WS haben wir grundsätzlich 2 Ziele Modelle kennengelernt:

\begin{itemize}
\tightlist
\item
  ``normale'' Regression (Werte + Bereiche)\\
\item
  logistische Regression (0 \textless-\textgreater{} 1)
\end{itemize}

Wichtiger Qualitätsfaktor: Zeit

Beispiel Auto: Wie lange habe ich, um zu erkennen, ob etwas vor mit ist
oder nicht? Muss ich erkennen, was vor mir ist? Wie detailliert?

Unterschiedlich für jedes UseCase

\hypertarget{fehlerberechnungen}{%
\subsection{Fehlerberechnungen}\label{fehlerberechnungen}}

\begin{itemize}
\tightlist
\item
  Konfidenzintervalle
\item
  Prädiktionsintervalle
\end{itemize}

Grenzverteilungssatz -\textgreater{} desto mehr Messungen n
-\textgreater{} desto kleiner wird der Fehler um den wahren Mittelwert

Bei Simulierungen wird die Anzahl n virtuell vervielfacht. Bringt
extreme \textbf{genauigkeit}.

\hypertarget{konfidenzintervalle-berechnung-in-r}{%
\subsubsection{Konfidenzintervalle berechnung in
R}\label{konfidenzintervalle-berechnung-in-r}}

\includegraphics[width=12cm, center]{Fig12}

\hypertarget{besipiel-haribo}{%
\subsubsection{Besipiel Haribo}\label{besipiel-haribo}}

\hypertarget{messfehler}{%
\subsubsection{Messfehler}\label{messfehler}}

Es gibt verschiedene Messfehler: - Stichprobengröße: Lässt sich
minimieren - Messfehler - x Range: Schmaler bereich -\textgreater{}
geringerer Fehler

\hypertarget{klassifikation-1}{%
\subsubsection{Klassifikation}\label{klassifikation-1}}

\begin{itemize}
\tightlist
\item
  True positive
\item
  True negative
\item
  False positive
\item
  False negative -\textgreater{} Type II Error -\textgreater{} Sehr
  schlecht und kann nicht so gut in den Griff bekommen werden.
\end{itemize}

\includegraphics[width=12cm, center]{Fig13}

Accuracy sagt vie wiele Beobachtungen man richtig erwischt erwischt.

Precision wie viele positives man als positive eingestuft hat.

\hypertarget{anpassung-von-modellen}{%
\subsubsection{Anpassung von Modellen}\label{anpassung-von-modellen}}

Numerische Optimierungsalgorithmen ändern das Outcome nicht wirklich. Es
ändern sich auf der 10-11 nachkommastelle Werte.

Resampling, (\textbf{ändern der Daten}) da kann sich schon mehr ändern
-\textgreater{} 4-5 nachkommastelle.

\begin{itemize}
\tightlist
\item
  Underfitting: zu ungenaues Modell
\item
  Overfitting: Man verkomliziert das Modell zu sehr.
\end{itemize}

Interpolation: Zieh eine Linie durch \textbf{alle} Punkte.

\includegraphics[width=12cm, center]{Fig14}

\includegraphics[width=12cm, center]{Fig15}\clearpage \clearpage

\appendix

\hypertarget{einfuxfchrung-in-r}{%
\section{Einführung in R}\label{einfuxfchrung-in-r}}

\hypertarget{operatoren-und-funktionen}{%
\subsection{Operatoren und Funktionen}\label{operatoren-und-funktionen}}

\hypertarget{arithmetische-operatoren}{%
\subsubsection{Arithmetische
Operatoren}\label{arithmetische-operatoren}}

Die Operatoren in R sind folgende:

\begin{Shaded}
\begin{Highlighting}[]
\SpecialCharTok{+}\NormalTok{               Addition}
\SpecialCharTok{{-}}\NormalTok{               Subtraktion}
\SpecialCharTok{*}\NormalTok{               Multiplikation}
\SpecialCharTok{/}\NormalTok{               Division}
\SpecialCharTok{\^{}}\NormalTok{ oder }\SpecialCharTok{**}\NormalTok{       Potenz}


\NormalTok{x }\SpecialCharTok{\%*\%}\NormalTok{ y         Matrixmultiplikation }\FunctionTok{c}\NormalTok{(}\DecValTok{5}\NormalTok{, }\DecValTok{3}\NormalTok{) }\SpecialCharTok{\%*\%} \FunctionTok{c}\NormalTok{(}\DecValTok{2}\NormalTok{, }\DecValTok{4}\NormalTok{) }\SpecialCharTok{==} \DecValTok{22}
\NormalTok{x }\SpecialCharTok{\%\%}\NormalTok{ y          }\FunctionTok{Modulo}\NormalTok{ (x mod y) }\DecValTok{5} \SpecialCharTok{\%\%} \DecValTok{2} \SpecialCharTok{==} \DecValTok{1}
\NormalTok{x }\SpecialCharTok{\%/\%}\NormalTok{ y         Ganzzahlige Teilung}\SpecialCharTok{:} \DecValTok{5} \SpecialCharTok{\%/\%} \DecValTok{2} \SpecialCharTok{==} \DecValTok{2}
\end{Highlighting}
\end{Shaded}

\hypertarget{logische-operatoren-und-funktionen}{%
\subsubsection{Logische Operatoren und
Funktionen}\label{logische-operatoren-und-funktionen}}

Für logische Operationen stehen einem folgende Operatoren zur Verfügung:

\begin{Shaded}
\begin{Highlighting}[]
\SpecialCharTok{\textless{}}\NormalTok{               Kleiner}
\SpecialCharTok{\textless{}=}\NormalTok{              Kleiner gleich}
\SpecialCharTok{\textgreater{}}\NormalTok{               Grösser}
\SpecialCharTok{\textgreater{}=}\NormalTok{              Grösser gleich}
\SpecialCharTok{==}              \FunctionTok{Gleich}\NormalTok{ (testet auf Äquivalenz)}
\SpecialCharTok{!=}\NormalTok{              Ungleich}
\SpecialCharTok{!}\NormalTok{x              Nicht }\FunctionTok{x}\NormalTok{ (Verneinung)}
\NormalTok{x }\SpecialCharTok{|}\NormalTok{ y           x ODER y}
\NormalTok{x }\SpecialCharTok{\&}\NormalTok{ y           x UND y}
\end{Highlighting}
\end{Shaded}

Außerdem gibt es noch zwei weitere Funktionen:

\begin{Shaded}
\begin{Highlighting}[]
\FunctionTok{xor}\NormalTok{(x, y)       Exklusives }\FunctionTok{ODER}\NormalTok{ (entweder }\ControlFlowTok{in}\NormalTok{ x oder y, aber nicht }\ControlFlowTok{in}\NormalTok{ beiden)}
\FunctionTok{isTRUE}\NormalTok{(x)       testet ob x wahr ist}
\end{Highlighting}
\end{Shaded}

\hypertarget{numerische-funktionen}{%
\subsubsection{Numerische Funktionen}\label{numerische-funktionen}}

Gewisse numerische Operationen sind in R als Funktionen ausgeführt:

\begin{Shaded}
\begin{Highlighting}[]
\FunctionTok{abs}\NormalTok{(x)                Betrag}
\FunctionTok{sqrt}\NormalTok{(x)               Quadratwurzel}
\FunctionTok{ceiling}\NormalTok{(x)            Aufrunden}\SpecialCharTok{:} \FunctionTok{ceiling}\NormalTok{(}\FloatTok{3.475}\NormalTok{) ist }\DecValTok{4}
\FunctionTok{floor}\NormalTok{(x)              Abrunden}\SpecialCharTok{:} \FunctionTok{floor}\NormalTok{(}\FloatTok{3.475}\NormalTok{) ist }\DecValTok{3}
\FunctionTok{round}\NormalTok{(x, }\AttributeTok{digits =}\NormalTok{ n)  Runden}\SpecialCharTok{:} \FunctionTok{round}\NormalTok{(}\FloatTok{3.475}\NormalTok{, }\AttributeTok{digits =} \DecValTok{2}\NormalTok{) ist }\FloatTok{3.48}
\FunctionTok{log}\NormalTok{(x)                Natürlicher Logarithmus}
\FunctionTok{log}\NormalTok{(x, }\AttributeTok{base =}\NormalTok{ n)      Logarithmus zur Basis n}
\FunctionTok{log2}\NormalTok{(x)               Logarithmus zur Basis }\DecValTok{2}
\FunctionTok{log10}\NormalTok{(x)              Logarithmus zur Basis }\DecValTok{10}
\FunctionTok{exp}\NormalTok{(x)                Exponentialfunktion}\SpecialCharTok{:}\NormalTok{ e}\SpecialCharTok{\^{}}\NormalTok{x}
\end{Highlighting}
\end{Shaded}

In R ist jeder Operator in Wirklichkeit eine Funktion, nur mit
spezieller Notation. Um sie in normaler Notation nützen zu können,
können sie mit Backticks umgeben werden:

\begin{Shaded}
\begin{Highlighting}[]
\DecValTok{2} \SpecialCharTok{+} \DecValTok{3}

\StringTok{\textasciigrave{}}\AttributeTok{+}\StringTok{\textasciigrave{}}\NormalTok{(}\DecValTok{2}\NormalTok{, }\DecValTok{3}\NormalTok{)}
\end{Highlighting}
\end{Shaded}

\hypertarget{r-als-taschenrechner}{%
\subsubsection{R als Taschenrechner}\label{r-als-taschenrechner}}

Ähnlich wie in Python kann man mit R ganz einfach rechnen.
Kommentarzeilen beginnen mit einem Hashtag (\texttt{\#}) und werden
nicht evaluiert:

\begin{Shaded}
\begin{Highlighting}[]
\CommentTok{\# Addition}
\DecValTok{5} \SpecialCharTok{+} \DecValTok{5}

\CommentTok{\# Subtraktion}
\DecValTok{6} \SpecialCharTok{{-}} \DecValTok{5}

\CommentTok{\# Multiplikation}
\DecValTok{34} \SpecialCharTok{*} \DecValTok{54}

\CommentTok{\# Division}
\DecValTok{4} \SpecialCharTok{/} \DecValTok{9}
\NormalTok{(}\DecValTok{5} \SpecialCharTok{+} \DecValTok{5}\NormalTok{) }\SpecialCharTok{/} \DecValTok{2}

\CommentTok{\# Klammernregel}
\DecValTok{1}\SpecialCharTok{/}\DecValTok{2} \SpecialCharTok{*}\NormalTok{ (}\DecValTok{12} \SpecialCharTok{+} \DecValTok{14} \SpecialCharTok{+} \DecValTok{10}\NormalTok{)}

\CommentTok{\# Potenzieren}
\DecValTok{3}\SpecialCharTok{\^{}}\DecValTok{2}
\FunctionTok{exp}\NormalTok{(}\DecValTok{5}\NormalTok{) }\CommentTok{\# geht auch mit der Exponentialfunktion}

\CommentTok{\# Ganzzahlige Division}
\CommentTok{\# 28 ist vier mal durch 6 teilbar, mit Rest 4 }
\DecValTok{28} \SpecialCharTok{\%\%} \DecValTok{6} \CommentTok{\# Rest: 4}
\DecValTok{28} \SpecialCharTok{\%/\%} \DecValTok{6} \CommentTok{\#  vier mal teilbar}

\CommentTok{\# Logische Operatoren}
\DecValTok{3} \SpecialCharTok{\textgreater{}} \DecValTok{2}
\DecValTok{4} \SpecialCharTok{\textgreater{}} \DecValTok{5}
\DecValTok{4} \SpecialCharTok{\textless{}} \DecValTok{4}
\DecValTok{4} \SpecialCharTok{\textless{}=} \DecValTok{4}
\DecValTok{5} \SpecialCharTok{\textgreater{}=} \DecValTok{5}
\DecValTok{6} \SpecialCharTok{!=} \DecValTok{6}
\DecValTok{9} \SpecialCharTok{==} \DecValTok{5} \SpecialCharTok{+} \DecValTok{4}

\SpecialCharTok{!}\NormalTok{(}\DecValTok{3} \SpecialCharTok{\textgreater{}} \DecValTok{2}\NormalTok{)}
\NormalTok{(}\DecValTok{3} \SpecialCharTok{\textgreater{}} \DecValTok{2}\NormalTok{) }\SpecialCharTok{\&}\NormalTok{ (}\DecValTok{4} \SpecialCharTok{\textgreater{}} \DecValTok{5}\NormalTok{) }\CommentTok{\# UND}
\NormalTok{(}\DecValTok{3} \SpecialCharTok{\textgreater{}} \DecValTok{2}\NormalTok{) }\SpecialCharTok{|}\NormalTok{ (}\DecValTok{4} \SpecialCharTok{\textgreater{}} \DecValTok{5}\NormalTok{) }\CommentTok{\# ODER}
\FunctionTok{xor}\NormalTok{((}\DecValTok{3} \SpecialCharTok{\textgreater{}} \DecValTok{2}\NormalTok{), (}\DecValTok{4} \SpecialCharTok{\textgreater{}} \DecValTok{5}\NormalTok{))}
\end{Highlighting}
\end{Shaded}

\hypertarget{statistische-funktionen}{%
\subsubsection{Statistische Funktionen}\label{statistische-funktionen}}

\begin{Shaded}
\begin{Highlighting}[]
\FunctionTok{mean}\NormalTok{(x, }\AttributeTok{na.rm =} \ConstantTok{FALSE}\NormalTok{)  Mittelwert}
\FunctionTok{sd}\NormalTok{(x)                   Standardabweichung}
\FunctionTok{var}\NormalTok{(x)                  Varianz}

\FunctionTok{median}\NormalTok{(x)               Median}

\FunctionTok{sum}\NormalTok{(x)                  Summe}
\FunctionTok{min}\NormalTok{(x)                  Minimalwert}
\FunctionTok{max}\NormalTok{(x)                  Maximalwert}
\FunctionTok{range}\NormalTok{(x)                Gibt Minimum und Maximum von x aus}
\end{Highlighting}
\end{Shaded}

\hypertarget{weitere-funktionen}{%
\subsubsection{Weitere Funktionen}\label{weitere-funktionen}}

\begin{Shaded}
\begin{Highlighting}[]
\FunctionTok{c}\NormalTok{()                     Erstellt }\FunctionTok{einen}\NormalTok{ (leeren) Vektor}
\FunctionTok{seq}\NormalTok{(from, to, by)       Generiert eine Sequenz}

\FunctionTok{rep}\NormalTok{(x, times, each)     Wiederholt x }
\NormalTok{                          times}\SpecialCharTok{:}\NormalTok{ die Sequenz wird n}\SpecialCharTok{{-}}\NormalTok{mal wiederholt}
\NormalTok{                          each}\SpecialCharTok{:}\NormalTok{ jedes Element wird n}\SpecialCharTok{{-}}\NormalTok{mal wiederholt}
                          
\FunctionTok{head}\NormalTok{(x, }\AttributeTok{n =} \DecValTok{6}\NormalTok{)          Gibt die ersten }\DecValTok{6}\NormalTok{ Elemente von x zurück}
\FunctionTok{tail}\NormalTok{(x, }\AttributeTok{n =} \DecValTok{6}\NormalTok{)          Gibt die letzten }\DecValTok{6}\NormalTok{ Elemente von x zurück}
\end{Highlighting}
\end{Shaded}

\hypertarget{variablen}{%
\subsection{Variablen}\label{variablen}}

\hypertarget{variablennamen}{%
\subsubsection{Variablennamen}\label{variablennamen}}

Beispiele:

\begin{Shaded}
\begin{Highlighting}[]
\CommentTok{\# Was gehen würde ...}
\NormalTok{snake\_case\_variable}
\NormalTok{camelCaseVariable}

\NormalTok{variable.with.periods}
\NormalTok{variable.With\_noConventions}

\NormalTok{x\_mean}
\NormalTok{x\_sd}

\NormalTok{anzahl\_personen}
\NormalTok{alter}

\NormalTok{p                       Nicht gut}
\NormalTok{a                       Nicht gut}

\CommentTok{\# und was nicht gehen würde:}
\NormalTok{x mittelwert}
\NormalTok{sd von x}
\end{Highlighting}
\end{Shaded}

Man kann in R sowohl \texttt{\textless{}-} - als auch \texttt{=} für die
Zuweisung verwenden.\texttt{=} ist ausserdem das Symbol für die
Zuweisung von Argumenten (für Funktionen), und wenn man
\texttt{\textless{}-} verwendet, ist es klar, dass man eine Variable
definiert.

\hypertarget{ausgeben-in-die-konsole}{%
\subsubsection{Ausgeben in die Konsole}\label{ausgeben-in-die-konsole}}

\begin{Shaded}
\begin{Highlighting}[]
\CommentTok{\# Unsere Variable:}
\NormalTok{my\_var }\OtherTok{\textless{}{-}} \DecValTok{4}

\CommentTok{\# Ausgeben mit:}
\FunctionTok{print}\NormalTok{(my\_var)}
\end{Highlighting}
\end{Shaded}

\begin{verbatim}
## [1] 4
\end{verbatim}

\begin{Shaded}
\begin{Highlighting}[]
\CommentTok{\# oder:}
\NormalTok{my\_var}
\end{Highlighting}
\end{Shaded}

\begin{verbatim}
## [1] 4
\end{verbatim}

\hypertarget{beispiel}{%
\subsubsection{Beispiel}\label{beispiel}}

\begin{Shaded}
\begin{Highlighting}[]
\NormalTok{vektor }\OtherTok{\textless{}{-}} \FunctionTok{c}\NormalTok{(}\DecValTok{1}\NormalTok{, }\DecValTok{3}\NormalTok{, }\DecValTok{4}\NormalTok{, }\DecValTok{7}\NormalTok{, }\DecValTok{11}\NormalTok{, }\DecValTok{2}\NormalTok{)}
\NormalTok{summe }\OtherTok{\textless{}{-}} \FunctionTok{sum}\NormalTok{(vektor)}

\NormalTok{mittelwert }\OtherTok{\textless{}{-}} \FunctionTok{mean}\NormalTok{(vektor)}
\NormalTok{mittelwert}

\NormalTok{gerundeter\_mittelwert }\OtherTok{\textless{}{-}} \FunctionTok{round}\NormalTok{(mittelwert, }\AttributeTok{digits =} \DecValTok{1}\NormalTok{)}
\NormalTok{gerundeter\_mittelwert}
\end{Highlighting}
\end{Shaded}

\hypertarget{funktionen}{%
\subsection{Funktionen}\label{funktionen}}

\hypertarget{beispiel-1}{%
\subsubsection{Beispiel}\label{beispiel-1}}

\begin{Shaded}
\begin{Highlighting}[]
\FunctionTok{function\_name}\NormalTok{(arg1, }\AttributeTok{arg2 =}\NormalTok{ val2)}
\end{Highlighting}
\end{Shaded}

\hypertarget{funktionsaufruf}{%
\subsubsection{Funktionsaufruf}\label{funktionsaufruf}}

\begin{Shaded}
\begin{Highlighting}[]
\NormalTok{x }\OtherTok{\textless{}{-}} \FunctionTok{c}\NormalTok{(}\FloatTok{23.192}\NormalTok{, }\FloatTok{21.454}\NormalTok{, }\FloatTok{24.677}\NormalTok{)}
\FunctionTok{round}\NormalTok{(x, }\AttributeTok{digits =} \DecValTok{1}\NormalTok{)}
\end{Highlighting}
\end{Shaded}

\begin{verbatim}
## [1] 23.2 21.5 24.7
\end{verbatim}

\hypertarget{default-werte}{%
\subsubsection{Default-Werte}\label{default-werte}}

Funktionen können Default-Werte haben; diese können beim Aufruf der
Funktion weggelassen werden. Zum Beispiel die Funktion \texttt{seq()}:

\begin{Shaded}
\begin{Highlighting}[]
\FunctionTok{seq}\NormalTok{()}
\FunctionTok{seq}\NormalTok{(}\DecValTok{1}\NormalTok{, }\DecValTok{10}\NormalTok{)}
\FunctionTok{seq}\NormalTok{(}\DecValTok{1}\NormalTok{, }\DecValTok{10}\NormalTok{, }\DecValTok{2}\NormalTok{)}
\FunctionTok{seq}\NormalTok{(}\DecValTok{1}\NormalTok{, }\DecValTok{10}\NormalTok{, }\DecValTok{2}\NormalTok{, }\DecValTok{20}\NormalTok{)}
\FunctionTok{seq}\NormalTok{(}\DecValTok{1}\NormalTok{, }\DecValTok{10}\NormalTok{, }\AttributeTok{length.out =} \DecValTok{20}\NormalTok{)}
\end{Highlighting}
\end{Shaded}

\hypertarget{datentypen}{%
\subsection{Datentypen}\label{datentypen}}

Der grundlegende Datentyp von R ist der Vektor. Alle anderen Datentypen
bauen auf diesem auf. Vektoren können in drei Typen geteilt werden:

\begin{itemize}
\tightlist
\item
  \textbf{numeric vectors}: Etwa ganze Zahlen \texttt{integer} und
  reelle Zahlen \texttt{double}.
\item
  \textbf{character vectors}: \ldots{} bestehen aus Zeichen, wie z.B.
  \texttt{"character"} und
  \texttt{\textquotesingle{}vectors\textquotesingle{}}.
\item
  \textbf{logical vectors}
\end{itemize}

Zusätzlich besitzen sie noch 3 Eigenschaften:

\begin{itemize}
\tightlist
\item
  Typ: \texttt{typeof()}
\item
  Länge: \texttt{length()}
\item
  Attribute: \texttt{attributes()}, sprich zusätzliche Informationen
  (Metadaten)
\end{itemize}

Vektoren können mit den oben gezeigten Methoden wie \texttt{c()} oder
mit speziellen Methoden wie \texttt{seq()} oder \texttt{rep()} erzeugt
werden.

\hypertarget{numeric-vectors}{%
\subsubsection{Numeric vectors}\label{numeric-vectors}}

\hypertarget{grundlegendes}{%
\paragraph{Grundlegendes}\label{grundlegendes}}

Numeric Vectors können entweder aus ganzen Zahlen oder reellen Zahlen
gebildet werden:

\begin{Shaded}
\begin{Highlighting}[]
\NormalTok{numbers }\OtherTok{\textless{}{-}} \FunctionTok{c}\NormalTok{(}\DecValTok{1}\NormalTok{, }\FloatTok{2.5}\NormalTok{, }\FloatTok{4.5}\NormalTok{)     Deklaration}
\FunctionTok{typeof}\NormalTok{(numbers)               Ausgabe des }\FunctionTok{Typs}\NormalTok{ (}\OtherTok{{-}\textgreater{}}\NormalTok{ double)}
\FunctionTok{length}\NormalTok{(numbers)               Ausgabe der Länge (}\OtherTok{{-}\textgreater{}} \DecValTok{3}\NormalTok{)}
\end{Highlighting}
\end{Shaded}

\hypertarget{zugriff}{%
\paragraph{Zugriff}\label{zugriff}}

Wir können die einzelnen Elemente eines Vektor mit {[}{]} auswählen:

\begin{Shaded}
\begin{Highlighting}[]
\NormalTok{numbers }\OtherTok{\textless{}{-}} \FunctionTok{c}\NormalTok{(}\DecValTok{1}\NormalTok{, }\FloatTok{2.5}\NormalTok{, }\FloatTok{4.5}\NormalTok{) }

\CommentTok{\# Das erste Element:}
\NormalTok{numbers[}\DecValTok{1}\NormalTok{]}
\end{Highlighting}
\end{Shaded}

\begin{verbatim}
## [1] 1
\end{verbatim}

\begin{Shaded}
\begin{Highlighting}[]
\CommentTok{\# Das zweite Element:}
\NormalTok{numbers[}\DecValTok{2}\NormalTok{]}
\end{Highlighting}
\end{Shaded}

\begin{verbatim}
## [1] 2.5
\end{verbatim}

\begin{Shaded}
\begin{Highlighting}[]
\CommentTok{\# Index des letzten Elements (von 1 ausgehend)}
\FunctionTok{length}\NormalTok{(numbers)}
\end{Highlighting}
\end{Shaded}

\begin{verbatim}
## [1] 3
\end{verbatim}

\begin{Shaded}
\begin{Highlighting}[]
\CommentTok{\# Zugriff auf das letzte Element}
\NormalTok{numbers[}\FunctionTok{length}\NormalTok{(numbers)]}
\end{Highlighting}
\end{Shaded}

\begin{verbatim}
## [1] 4.5
\end{verbatim}

\begin{Shaded}
\begin{Highlighting}[]
\CommentTok{\# Mit {-} (Minus) können wir ein Element weglassen:}
\NormalTok{numbers[}\SpecialCharTok{{-}}\DecValTok{1}\NormalTok{]}
\end{Highlighting}
\end{Shaded}

\begin{verbatim}
## [1] 2.5 4.5
\end{verbatim}

\begin{Shaded}
\begin{Highlighting}[]
\CommentTok{\# Bilden einer Teilsequenz aus numbers:}
\NormalTok{numbers[}\DecValTok{1}\SpecialCharTok{:}\DecValTok{2}\NormalTok{]}
\end{Highlighting}
\end{Shaded}

\begin{verbatim}
## [1] 1.0 2.5
\end{verbatim}

\begin{Shaded}
\begin{Highlighting}[]
\CommentTok{\# Weglassen des ersten und dritten Elements weglassen:}
\NormalTok{numbers[}\SpecialCharTok{{-}}\FunctionTok{c}\NormalTok{(}\DecValTok{1}\NormalTok{, }\DecValTok{3}\NormalTok{)]}
\end{Highlighting}
\end{Shaded}

\begin{verbatim}
## [1] 2.5
\end{verbatim}

\hypertarget{matrizen}{%
\paragraph{Matrizen}\label{matrizen}}

Matrizen sind spezielle Vektoren:

\begin{Shaded}
\begin{Highlighting}[]
\NormalTok{x }\OtherTok{\textless{}{-}} \DecValTok{1}\SpecialCharTok{:}\DecValTok{8}

\CommentTok{\# Bilden der Matrix durch Zuweisen des dim{-}Attributes:}
\FunctionTok{dim}\NormalTok{(x) }\OtherTok{\textless{}{-}} \FunctionTok{c}\NormalTok{(}\DecValTok{2}\NormalTok{, }\DecValTok{4}\NormalTok{)}

\CommentTok{\# oder mit:}
\NormalTok{m }\OtherTok{\textless{}{-}} \FunctionTok{matrix}\NormalTok{(}\DecValTok{1}\SpecialCharTok{:}\DecValTok{8}\NormalTok{, }\AttributeTok{nrow =} \DecValTok{2}\NormalTok{, }\AttributeTok{ncol =} \DecValTok{4}\NormalTok{, }\AttributeTok{byrow =} \ConstantTok{FALSE}\NormalTok{)}
\NormalTok{m}
\end{Highlighting}
\end{Shaded}

\begin{verbatim}
##      [,1] [,2] [,3] [,4]
## [1,]    1    3    5    7
## [2,]    2    4    6    8
\end{verbatim}

\begin{Shaded}
\begin{Highlighting}[]
\CommentTok{\# byrow ändert, wie die Matrix gefüllt wird (standardmäßig FALSE)}
\NormalTok{m2 }\OtherTok{\textless{}{-}} \FunctionTok{matrix}\NormalTok{(}\DecValTok{1}\SpecialCharTok{:}\DecValTok{8}\NormalTok{, }\AttributeTok{nrow =} \DecValTok{2}\NormalTok{, }\AttributeTok{ncol =} \DecValTok{4}\NormalTok{, }\AttributeTok{byrow =} \ConstantTok{TRUE}\NormalTok{)}
\NormalTok{m2}
\end{Highlighting}
\end{Shaded}

\begin{verbatim}
##      [,1] [,2] [,3] [,4]
## [1,]    1    2    3    4
## [2,]    5    6    7    8
\end{verbatim}

\begin{Shaded}
\begin{Highlighting}[]
\CommentTok{\# Transponieren:}
\NormalTok{(m\_transponiert }\OtherTok{\textless{}{-}} \FunctionTok{t}\NormalTok{(m))}
\end{Highlighting}
\end{Shaded}

\begin{verbatim}
##      [,1] [,2]
## [1,]    1    2
## [2,]    3    4
## [3,]    5    6
## [4,]    7    8
\end{verbatim}

Indizieren:

\begin{Shaded}
\begin{Highlighting}[]
\NormalTok{m1[}\DecValTok{1}\NormalTok{, }\DecValTok{1}\NormalTok{]                Zeile }\DecValTok{1}\NormalTok{, Spalte }\DecValTok{1}
\NormalTok{m1[}\DecValTok{1}\NormalTok{, }\DecValTok{2}\NormalTok{]                Zeile }\DecValTok{1}\NormalTok{, Spalte }\DecValTok{2}
\NormalTok{m1[}\DecValTok{2}\SpecialCharTok{:}\DecValTok{3}\NormalTok{, }\DecValTok{1}\NormalTok{]              Zeilen }\DecValTok{2}\NormalTok{ bis }\DecValTok{3}\NormalTok{, Spalte }\DecValTok{1}

\NormalTok{m1[, }\DecValTok{1}\NormalTok{]                 Alle Zeilen, Spalte }\DecValTok{1}
\NormalTok{m1[}\DecValTok{2}\NormalTok{, ]                 Zeile }\DecValTok{2}\NormalTok{, Alle Spalten}
\end{Highlighting}
\end{Shaded}

\hypertarget{vektorisierung}{%
\paragraph{Vektorisierung}\label{vektorisierung}}

Alle Operatoren wirken elementweise auf Vektoren:

\begin{Shaded}
\begin{Highlighting}[]
\NormalTok{x1 }\OtherTok{\textless{}{-}} \DecValTok{1}\SpecialCharTok{:}\DecValTok{10}
\NormalTok{x2 }\OtherTok{\textless{}{-}} \DecValTok{11}\SpecialCharTok{:}\DecValTok{20}

\NormalTok{x1 }\SpecialCharTok{+} \DecValTok{2}
\CommentTok{\#\textgreater{} [1] 2 3}

\NormalTok{x1 }\SpecialCharTok{+}\NormalTok{ x2}
\CommentTok{\#\textgreater{} [1] 1 2 3}

\NormalTok{x1 }\SpecialCharTok{*}\NormalTok{ x2}
\CommentTok{\#\textgreater{} x1 x2 }
\CommentTok{\#\textgreater{}  2 11}
\end{Highlighting}
\end{Shaded}

Dasselbe gilt für Funktionen:

\begin{Shaded}
\begin{Highlighting}[]
\NormalTok{x1 }\OtherTok{\textless{}{-}} \DecValTok{1}\SpecialCharTok{:}\DecValTok{10}

\NormalTok{x1}\SpecialCharTok{\^{}}\DecValTok{2}
\CommentTok{\#\textgreater{}  [1]   1   4   9  16  25  36  49  64  81 100}

\FunctionTok{exp}\NormalTok{(x1)}
\CommentTok{\#\textgreater{}  [1]     2.718282     7.389056    20.085537    54.598150   148.413159}
\CommentTok{\#\textgreater{}  [6]   403.428793  1096.633158  2980.957987  8103.083928 22026.465795}
\end{Highlighting}
\end{Shaded}

\hypertarget{missing-values}{%
\paragraph{Missing Values}\label{missing-values}}

Fehlende Werte werden mit \texttt{NA} deklariert.

\begin{Shaded}
\begin{Highlighting}[]
\NormalTok{zahlen }\OtherTok{\textless{}{-}} \FunctionTok{c}\NormalTok{(}\DecValTok{12}\NormalTok{, }\DecValTok{13}\NormalTok{, }\DecValTok{15}\NormalTok{, }\DecValTok{11}\NormalTok{, }\ConstantTok{NA}\NormalTok{, }\DecValTok{10}\NormalTok{)}
\NormalTok{zahlen}
\CommentTok{\#\textgreater{} [1] 12 13 15 11 NA 10}
\end{Highlighting}
\end{Shaded}

Überprüfen mit \texttt{is.na()}:

\begin{Shaded}
\begin{Highlighting}[]
\FunctionTok{is.na}\NormalTok{(zahlen)}
\CommentTok{\#\textgreater{} [1] FALSE FALSE FALSE FALSE  TRUE FALSE}
\end{Highlighting}
\end{Shaded}

\hypertarget{character-vectors}{%
\subsubsection{Character Vectors}\label{character-vectors}}

\hypertarget{grundlegendes-1}{%
\paragraph{Grundlegendes}\label{grundlegendes-1}}

Character vectors (strings) werden benützt, um Text darzustellen.

\begin{Shaded}
\begin{Highlighting}[]
\NormalTok{(text }\OtherTok{\textless{}{-}} \FunctionTok{c}\NormalTok{(}\StringTok{"these are"}\NormalTok{, }\StringTok{"some strings"}\NormalTok{))}

\CommentTok{\#\textgreater{} [1] "these are"    "some strings"}

\FunctionTok{typeof}\NormalTok{(text)}
\CommentTok{\#\textgreater{} [1] "character"}

\FunctionTok{length}\NormalTok{(text)}
\CommentTok{\#\textgreater{} [1] 2}
\end{Highlighting}
\end{Shaded}

\hypertarget{zusammenfuxfcgen-von-character-vectors}{%
\paragraph{Zusammenfügen von Character
vectors}\label{zusammenfuxfcgen-von-character-vectors}}

\begin{Shaded}
\begin{Highlighting}[]
\NormalTok{vorname }\OtherTok{\textless{}{-}} \StringTok{"Andreas"}
\NormalTok{nachname }\OtherTok{\textless{}{-}} \StringTok{"Sünder"}
\FunctionTok{paste}\NormalTok{(}\StringTok{"Mein Name ist:"}\NormalTok{, vorname, nachname, }\AttributeTok{sep =} \StringTok{" "}\NormalTok{)}
\CommentTok{\#\textgreater{} [1] "Mein Name ist: Andreas Sünder"}
\end{Highlighting}
\end{Shaded}

\hypertarget{logical-vectors}{%
\subsubsection{Logical vectors}\label{logical-vectors}}

Bei Logical vectors sind drei Werte möglich: \texttt{TRUE},
\texttt{FALSE} oder \texttt{NA}.

\begin{Shaded}
\begin{Highlighting}[]
\NormalTok{log\_var }\OtherTok{\textless{}{-}} \FunctionTok{c}\NormalTok{(}\ConstantTok{TRUE}\NormalTok{, }\ConstantTok{FALSE}\NormalTok{, }\ConstantTok{TRUE}\NormalTok{)}
\CommentTok{\#\textgreater{} [1]  TRUE FALSE  TRUE}
\end{Highlighting}
\end{Shaded}

\hypertarget{factors}{%
\subsubsection{Factors}\label{factors}}

Ein factor ist ein Vektor von natürlichen Zahlen (integer vector), der
mit zusätzlicher Information (Metadaten) versehen ist. Diese
\texttt{attributes} sind die Objektklasse \texttt{factor} und die
Faktorstufen \texttt{levels.} Ein Beispiel:

\begin{Shaded}
\begin{Highlighting}[]
\NormalTok{geschlecht }\OtherTok{\textless{}{-}} \FunctionTok{c}\NormalTok{(}\StringTok{"male"}\NormalTok{, }\StringTok{"female"}\NormalTok{, }\StringTok{"male"}\NormalTok{, }\StringTok{"male"}\NormalTok{, }\StringTok{"female"}\NormalTok{)}
\CommentTok{\#\textgreater{} [1] "male"   "female" "male"   "male"   "female"}

\NormalTok{geschlecht }\OtherTok{\textless{}{-}} \FunctionTok{factor}\NormalTok{(geschlecht, }\AttributeTok{levels =} \FunctionTok{c}\NormalTok{(}\StringTok{"female"}\NormalTok{, }\StringTok{"male"}\NormalTok{))}

\FunctionTok{table}\NormalTok{(geschlecht)}
\CommentTok{\#\textgreater{} geschlecht}
\CommentTok{\#\textgreater{}   male female }
\CommentTok{\#\textgreater{}      3      2}
\end{Highlighting}
\end{Shaded}

\hypertarget{listen}{%
\subsubsection{Listen}\label{listen}}

\hypertarget{grundlegendes-2}{%
\paragraph{Grundlegendes}\label{grundlegendes-2}}

Anders als bei Vektoren müssen bei Listen die Elemente nicht denselben
Datentyp besitzen:

\begin{Shaded}
\begin{Highlighting}[]
\NormalTok{x }\OtherTok{\textless{}{-}} \FunctionTok{list}\NormalTok{(}\DecValTok{1}\SpecialCharTok{:}\DecValTok{3}\NormalTok{, }\StringTok{"a"}\NormalTok{, }\FunctionTok{c}\NormalTok{(}\ConstantTok{TRUE}\NormalTok{, }\ConstantTok{FALSE}\NormalTok{, }\ConstantTok{TRUE}\NormalTok{), }\FunctionTok{c}\NormalTok{(}\FloatTok{2.3}\NormalTok{, }\FloatTok{5.9}\NormalTok{))}
\CommentTok{\#\textgreater{} [[1]]}
\CommentTok{\#\textgreater{} [1] 1 2 3}
\CommentTok{\#\textgreater{} }
\CommentTok{\#\textgreater{} [[2]]}
\CommentTok{\#\textgreater{} [1] "a"}
\CommentTok{\#\textgreater{} }
\CommentTok{\#\textgreater{} [[3]]}
\CommentTok{\#\textgreater{} [1]  TRUE FALSE  TRUE}
\CommentTok{\#\textgreater{} }
\CommentTok{\#\textgreater{} [[4]]}
\CommentTok{\#\textgreater{} [1] 2.3 5.9}

\FunctionTok{typeof}\NormalTok{(x)}
\CommentTok{\#\textgreater{} [1] "list"}
\end{Highlighting}
\end{Shaded}

Listen können ebenfalls indiziert werden:

\begin{Shaded}
\begin{Highlighting}[]
\NormalTok{x[}\DecValTok{1}\NormalTok{]}
\CommentTok{\#\textgreater{} [[1]]}
\CommentTok{\#\textgreater{} [1] 1 2 3}
\NormalTok{x[}\DecValTok{2}\NormalTok{]}
\CommentTok{\#\textgreater{} [[1]]}
\CommentTok{\#\textgreater{} [1] "a"}
\end{Highlighting}
\end{Shaded}

\hypertarget{named-lists}{%
\paragraph{Named lists}\label{named-lists}}

Viele statistische Funktionen liefern eine \texttt{named\ list} als
Output:

\begin{Shaded}
\begin{Highlighting}[]
\NormalTok{x }\OtherTok{\textless{}{-}} \FunctionTok{list}\NormalTok{(}\AttributeTok{int =} \DecValTok{1}\SpecialCharTok{:}\DecValTok{3}\NormalTok{,}
          \AttributeTok{string =} \StringTok{"a"}\NormalTok{,}
          \AttributeTok{log =} \FunctionTok{c}\NormalTok{(}\ConstantTok{TRUE}\NormalTok{, }\ConstantTok{FALSE}\NormalTok{, }\ConstantTok{TRUE}\NormalTok{),}
          \AttributeTok{double =} \FunctionTok{c}\NormalTok{(}\FloatTok{2.3}\NormalTok{, }\FloatTok{5.9}\NormalTok{))}
\CommentTok{\#\textgreater{} $int}
\CommentTok{\#\textgreater{} [1] 1 2 3}
\CommentTok{\#\textgreater{} }
\CommentTok{\#\textgreater{} $string}
\CommentTok{\#\textgreater{} [1] "a"}
\CommentTok{\#\textgreater{} }
\CommentTok{\#\textgreater{} $log}
\CommentTok{\#\textgreater{} [1]  TRUE FALSE  TRUE}
\CommentTok{\#\textgreater{} }
\CommentTok{\#\textgreater{} $double}
\CommentTok{\#\textgreater{} [1] 2.3 5.9}
\end{Highlighting}
\end{Shaded}

Die Elemente können nun mit ihrem Namen über den \texttt{\$} Operator
angesprochen werden:

\begin{Shaded}
\begin{Highlighting}[]
\NormalTok{x}\SpecialCharTok{$}\NormalTok{string}
\CommentTok{\#\textgreater{} [1] "a"}
\NormalTok{x}\SpecialCharTok{$}\NormalTok{double}
\CommentTok{\#\textgreater{} [1] 2.3 5.9}
\end{Highlighting}
\end{Shaded}

\hypertarget{data-frames}{%
\subsubsection{Data Frames}\label{data-frames}}

\hypertarget{grundlegendes-3}{%
\paragraph{Grundlegendes}\label{grundlegendes-3}}

Ein Data Frame besteht aus Zeilen (rows) und Spalten (columns) und
entspricht einem Datensatz. Technisch gesehen ist ein Data Frame eine
Liste, deren Elemente gleichlange (equal-length) Vektoren sind. Die
Vektoren selber können numeric, logical oder character Vektoren sein,
oder natürlich Faktoren. Data Frame ist eine 2-dimensionale Struktur,
und kann einerseits wie ein Vektor indiziert werden (genauer: wie eine
Matrix), andererseits wie eine Liste.

Data Frames werden in R mitels \texttt{data.frame()} erstellt. In
RStudio bzw. dem \texttt{tidyverse}-Package werden Data Frames auch
\texttt{tibbles} genannt. \texttt{tibbles} werden mit der Funktion
\texttt{tibble()} erstellt, und stellen lediglich eine moderne Variante
eines Data Frames dar.

\begin{Shaded}
\begin{Highlighting}[]
\FunctionTok{library}\NormalTok{(dplyr)}

\NormalTok{df }\OtherTok{\textless{}{-}} \FunctionTok{tibble}\NormalTok{(}\AttributeTok{geschlecht =} \FunctionTok{factor}\NormalTok{(}\FunctionTok{c}\NormalTok{(}\StringTok{"male"}\NormalTok{, }\StringTok{"female"}\NormalTok{,}
                                       \StringTok{"male"}\NormalTok{, }\StringTok{"male"}\NormalTok{,}
                                       \StringTok{"female"}\NormalTok{)),}
                 \AttributeTok{alter =} \FunctionTok{c}\NormalTok{(}\DecValTok{22}\NormalTok{, }\DecValTok{45}\NormalTok{, }\DecValTok{33}\NormalTok{, }\DecValTok{27}\NormalTok{, }\DecValTok{30}\NormalTok{))}
\NormalTok{df}
\CommentTok{\#\textgreater{} \# A tibble: 5 × 2}
\CommentTok{\#\textgreater{}   geschlecht alter}
\CommentTok{\#\textgreater{}   \textless{}fct\textgreater{}      \textless{}dbl\textgreater{}}
\CommentTok{\#\textgreater{} 1 male          22}
\CommentTok{\#\textgreater{} 2 female        45}
\CommentTok{\#\textgreater{} 3 male          33}
\CommentTok{\#\textgreater{} 4 male          27}
\CommentTok{\#\textgreater{} 5 female        30}
\end{Highlighting}
\end{Shaded}

Ein Data Frame hat die Attribute \texttt{names()} (dasselbe wie
\texttt{colnames()}) und \texttt{rownames()}. Die Länge eines Data
Frames ist die Länge der Liste, d.h. sie entspricht der Anzahl der
Spalten. Diese kann mit \texttt{ncol()} abgefragt werden;
\texttt{nrow()} gibt die Anzahl der Zeilen.

\begin{Shaded}
\begin{Highlighting}[]
\FunctionTok{ncol}\NormalTok{(df)}
\CommentTok{\#\textgreater{} [1] 2}
\FunctionTok{nrow}\NormalTok{(df)}
\CommentTok{\#\textgreater{} [1] 5}
\end{Highlighting}
\end{Shaded}

\hypertarget{data-frame-subsetting}{%
\paragraph{Data Frame Subsetting}\label{data-frame-subsetting}}

Ein Data Frame kann wie eine Liste (mittels \texttt{\$}) oder wie eine
Matrix (mittels \texttt{{[}{]}}) indiziert werden:

\begin{Shaded}
\begin{Highlighting}[]
\NormalTok{df}\SpecialCharTok{$}\NormalTok{geschlecht}
\CommentTok{\#\textgreater{} [1] male   female male   male   female}
\CommentTok{\#\textgreater{} Levels: female male}

\NormalTok{df[}\StringTok{"alter"}\NormalTok{]}
\CommentTok{\#\textgreater{} \# A tibble: 5 × 1}
\CommentTok{\#\textgreater{}   alter}
\CommentTok{\#\textgreater{}   \textless{}dbl\textgreater{}}
\CommentTok{\#\textgreater{} 1    22}
\CommentTok{\#\textgreater{} 2    45}
\CommentTok{\#\textgreater{} 3    33}
\CommentTok{\#\textgreater{} 4    27}
\CommentTok{\#\textgreater{} 5    30}

\NormalTok{df[}\DecValTok{1}\NormalTok{]}
\CommentTok{\#\textgreater{} \# A tibble: 5 × 1}
\CommentTok{\#\textgreater{}   geschlecht}
\CommentTok{\#\textgreater{}   \textless{}fct\textgreater{}     }
\CommentTok{\#\textgreater{} 1 male      }
\CommentTok{\#\textgreater{} 2 female    }
\CommentTok{\#\textgreater{} 3 male      }
\CommentTok{\#\textgreater{} 4 male      }
\CommentTok{\#\textgreater{} 5 female}

\NormalTok{df[}\DecValTok{1}\NormalTok{, }\DecValTok{1}\NormalTok{]}
\CommentTok{\#\textgreater{} \# A tibble: 1 × 1}
\CommentTok{\#\textgreater{}   geschlecht}
\CommentTok{\#\textgreater{}   \textless{}fct\textgreater{}     }
\CommentTok{\#\textgreater{} 1 male}

\NormalTok{df[ , ]}
\CommentTok{\#\textgreater{} \# A tibble: 5 × 2}
\CommentTok{\#\textgreater{}   geschlecht alter}
\CommentTok{\#\textgreater{}   \textless{}fct\textgreater{}      \textless{}dbl\textgreater{}}
\CommentTok{\#\textgreater{} 1 male          22}
\CommentTok{\#\textgreater{} 2 female        45}
\CommentTok{\#\textgreater{} 3 male          33}
\CommentTok{\#\textgreater{} 4 male          27}
\CommentTok{\#\textgreater{} 5 female        30}

\NormalTok{df[}\DecValTok{1}\SpecialCharTok{:}\DecValTok{3}\NormalTok{, ]}
\CommentTok{\#\textgreater{} \# A tibble: 3 × 2}
\CommentTok{\#\textgreater{}   geschlecht alter}
\CommentTok{\#\textgreater{}   \textless{}fct\textgreater{}      \textless{}dbl\textgreater{}}
\CommentTok{\#\textgreater{} 1 male          22}
\CommentTok{\#\textgreater{} 2 female        45}
\CommentTok{\#\textgreater{} 3 male          33}
\end{Highlighting}
\end{Shaded}

Da die Spalten ebenfalls Vektoren sind, kann man diese auch indizieren:

\begin{Shaded}
\begin{Highlighting}[]
\NormalTok{df}\SpecialCharTok{$}\NormalTok{geschlecht[}\DecValTok{1}\NormalTok{]}
\CommentTok{\#\textgreater{} [1] male}
\CommentTok{\#\textgreater{} Levels: female male}

\NormalTok{df}\SpecialCharTok{$}\NormalTok{alter[}\DecValTok{2}\SpecialCharTok{:}\DecValTok{3}\NormalTok{]}
\CommentTok{\#\textgreater{} [1] 45 33}
\end{Highlighting}
\end{Shaded}


\end{document}
